Dette kandidat speciale omhandler udbyggende arbejde på AAUSHIP platformen. AAUSHIP er et \ac{ASV} som kan have flere forskellige anvendelsesmuligheder. Fokus i dette project vil være omhandlende pejlingsopgaver hvor AAUSHIP skal udvides fra et enkelt skib til en flåde af flere AAUSHIPs, som skal navigere i formation.

Båden er blevet fysisk blevet opgraderet med ting som nyt Wi-Fi model osv. som er implementeret. Dertil er den enkelte båd blevet testet med et nyt \ac{KF} hvortil en retningsregulator også er implementeret. Denne er anvendt som en \ac{LOS} reference regulator for at få båden til at konvergere til de genererede liniestykker mellem rutepunkter. Efterfølgende har fokus været sat på at identificere og analysere formationsstrategier som skal implementeres på bådene, når disse er produceret. Mest fokus er lagt på en potentialefeltsalgoritme, som er blevet simuleret hvilken inkluderer dynamikken fra AAUSHIP. Dette danner grundlaget for at implementering på flåden af både med opfølgende verifikation.

Resultater viser at det er muligt at kontrollere AAUSHIP med den designede model in et område givet fra Aalborg Havn. Yderligere arbejde vil ligge i at forbedre modellen af AAUSHIP, men dette har ikke vist sig at være nødvendigt da focus i dette project har omhandlet pejling sopgaven. Slutteligt udviser simulering af formationskontrolstrategien potentiale for en mulig implementering af denne på den kommende flåde af AAUSHIPs, så disse kan foretage pejligsopgaver som en dannet formation af skibe.