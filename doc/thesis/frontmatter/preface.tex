\chapter{Preface}
\todo[inline]{This has to be written some time in the future, describe what this doc is.}

\subsubsection*{Thanks to}
\todo[inline]{Say thanks to someone, Aalborg Havn if they are helpfull?}


\begin{center}
  \begin{minipage}[t]{0.47\textwidth}
    \centering \vspace{1.5cm} \hrule \vspace{1mm} Nick \O stergaard
  \end{minipage}
  \hfill
  \begin{minipage}[t]{0.47\textwidth}
    \centering \vspace{1.5cm} \hrule \vspace{1mm} Jeppe Dam
  \end{minipage}
\end{center}


\newpage
\section{Reading guide}
The following report is divided into parts, related to different phases of the project. The parts are divided into chapters, the chapters describe different aspects of the project. The chapters are subdivided a number of times to further split up the content into specific topics. The report is ended with an appendix part, that contains all the material that is relevant to the project, but not necessarily interesting to the reader, such as measurement journals and transcripts of meetings.

\begin{description}
\item[Citations] in the report is done according to the Harvard method, the list of references can be found \vpageref{ch:litt}. The elements on the list of references are sorted by author.
\item[Acronyms] are written to their full extend, the first time they are used, with the acronym in parentheses, thereafter only the acronym is used. The list of acronyms can be found \vpageref{ch:acronyms}.
\item[Notation] of vectors are written in bold font with lower case letters ($\vec{v}$), matrices are written in bold font with upper case letters ($\vec{M}$). Single variables and constants are typeset in normal math ($x$).
\item[Attached] to the report is a CD, which contains copies of web references and other digital files (source code, scripts and raw measurement data) that could be of interest to the reader. In some places in the report there will be a reference to the CD; this will look like this: \cd{/path-to-file}.
\end{description}
