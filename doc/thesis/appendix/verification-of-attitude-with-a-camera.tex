\chapter{Verification of attitude with a camera}
\head{This is a description of a technique to verify an attitude
estimator by using a camera. It utilizes the horizon as the reference.}

\section{Objective}
\section{Method}


While testing it is of importance to make some verifications. When implementing an attitude
estimator for pitch and roll, it is hard to verify that it works as
intended. A simple way to test it is to put the sensor in positions
that can be measured in a static environment, that is by i.e. putting
in on a table on what is to be defined as flat, then angle it
some known degrees and see if the estimator agrees.

Depending on how exotic the estimator is, it might utilise a dynamic
model, in an attempt to improve the accuracy of the estimates. This is
harder to measure in the static lab setup. Another way is to record
the attitude with another setup that is known to work, but that could
not exist, hence the need to implement a new one. Alternatively a
camera can be used, and this is the method to be described in the
following.

One could also use a visual means of determining the heading. The idea
here is to mount a camera on the rigid body object containing the
sensor in the longitudinal direction of the axis of interest. That is
a camera pointing forward for the roll determiniaton or a camera
pointing to one side for the pitch determiniaton. This is intended for
use on ships.

This method relies on a known reference, that is stationary or at least
known at all time. The most ideal scenario is to be in open water
where the horizon is between the sea and the sky. This is guaranteed
to be horizontal, giving us an absolute reference to compare against.

In a non ideal scenario is to e.g. use the harbour quay. This works if
the ship is not moving a lot around relative to the quay, else you
need to know the exact position to the quay to determine the angle the
camera should see as zero. This is also possible but is out of scope
of this description.

\missingfigure{Screenshot of the images generated}
