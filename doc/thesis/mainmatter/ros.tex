\chapter{\acs{ROS} Design}
\head{This chapter describes the details of the implementation level,
which is done on a \acl{ROS} platform. This should provide all the
information needed to complete the implementation.}

It was decided to use \ac{ROS} in the implementation. It is used as an
other abstraction layer on top of the \ac{LLI}. In turn it makes AAUSHIP
modular and make it easy for others to write parts of the control
system without reimplementing basic components. This in turn makes it
an extensible platform, that should be easy to extend. \ac{ROS} is a
project available at \url{http://ros.org}, which describes itself in
short as following:

\begin{quote}
\textit{\noindent
	ROS (Robot Operating System) provides libraries and
	tools to help software developers create robot applications. It
	provides hardware abstraction, device drivers, libraries,
	visualizers, message-passing, package management, and more. ROS is
	licensed under an open source, BSD license.
}
		
	\hfill ROS.org
\end{quote}

\section{\acs{ROS} Terminology}
To start working with \ac{ROS} it is important to use the terminology
used by \ac{ROS} to avoid confusion. Therefore these  will be stated
in this section.  The idea of \ac{ROS} is to make it easy to build a
system modularly, and this is achieved by using almost
``self-contained'' code segments called \textit{nodes}, which is
application parts that is run as its own process. A node should be
designed to execute limited tasks such as image processing or similar
atomic processes. These nodes can then communicate with other nodes by
the means of two main communication forms called \textit{topics} and
\textit{services}. A good introduction to the base components of \ac{ROS}
and overview of how it works is also described by the
\citep{rosoverview} conference paper.

\begin{description}
\item[The topic] is an asynchronous connection, that can \textit{publish}
from many nodes and be \textit{subscribed} by many nodes. This means
that it is a multicast form for providing data.
\item[The service] is a synchronous connection that is used between one node
to another node. This is only unicast.
\end{description}

An illustration of multiple nodes connected via topics and a service
is on figure~\vref{fig:ros-node-simple-concept}. This concept can also
be used across multiple machines. This is illustrated in
figure~\vref{fig:ros-node-master-concept}. On this a new type of unique
node is introduced, this is the \ac{ROS} master. This is a required
component for a ROS system to run. The master's only purpose is to make
the nodes connect together via the topics or services. There can only
be one master per \ac{ROS} system. This also enables multiple machines
to share topics, by connecting the masters that runs on the individual machines. In
short the masters sole purpose is to make these connections. It is
illustrated by the dashed arrows. Each node says that it want to i.e.
publish or subscribe to a certain topic. To connect a node from one
machine to another, the environment variable \texttt{ROS\_MASTER\_URI}
has to be set to the host with the master running, and the
\texttt{/etc/hosts} file has to be set on all machines with the other
machine's hostnames.

\begin{figure}[htbp]
	\centering
	\includesvg{ros_node_simple_concept}
	\caption{Basic principle of the node abstraction illustrating a
	service and two topics. The topology chosen here is only to illustrate
	the possibilities.}
	\label{fig:ros-node-simple-concept}
\end{figure}

When that is said, that is not the whole picture of the topology. In a
need to make this flexible \ac{ROS} has made it such that the nodes
can be started and stopped kind of ``runtime''. That is such that it is
possible to have different configurations of nodes to run in different
scenarios, i.e. in development with debugging nodes and virtual sensor
nodes versus in the real mission where no debugging nodes is used and
real sensor nodes that use real sensor data is used.

\begin{figure}[htbp]
	\centering
	\includesvg{ros_node_master_concept}
	\caption{Concept showing the ROS master together with the nodes,
	also illustrating the masters role with multiple machines. Dashed
	lines hows that the node will either subscribe or publish to the
	topic. This only happens initially when connecting to a topic. Gray
	area is two physical separate but networked machines.}
	\label{fig:ros-node-master-concept}
\end{figure}

\section{ROS on AAUSHIP}

\begin{description}
	\item[\texttt{/lli\_node}]\hfill \\
		This is responsible for communicating with the \ac{LLI}. It
		publishes sensor data from the \ac{LLI} and receives various input
		commands and forwards them to the \ac{LLI}.
	\item[\texttt{/sensor\_decode\_node}]\hfill \\
		This decodes the samples from the \ac{LLI} and publishes this to
		the relevant topics. 
	\item[\texttt{/joy}]\hfill \\
		This is a node from \ac{ROS} that is a generic interface to
		various joystick inputs. Here it is used to get data from a PS3
		controller.
	\item[\texttt{/joy\_teleop\_node}]\hfill \\
		This is node parses some data from the PlayStation 3 controller, and formats
		those as messages that is to be sent to the \ac{LLI}.
	\item[\texttt{/kf\_node}]\hfill \\
		This node is the running the \ac{KF} iteration with the attitude estimate and
		the decoded \ac{IMU} samples.
	\item[\texttt{/gnc\_node}]\hfill \\
		This node runs the \ac{GNC} algorithm
	\item[\texttt{/rqt\_mission\_planner\_node}]\hfill \\
		This node is an rqt plugin used to specify the mission.
\end{description}

\begin{figure}[htbp]
	\centering
	\includesvg{ros_aauship_teleop}
	\caption{ROS configuration on AAUSHIP for manual tele operation.
	AAUSHIP and GRS are connected via a Wi-Fi connection.}
	\label{fig:ros-aauship-teleop}
\end{figure}

\begin{figure}[htbp]
	\centering
	{\footnotesize
	\includesvg{ros_aauship_closed_loop_single}}
	\caption{Closed loop ROS system on one ship.}
	\label{fig:ros-aauship-closed-loop-single}
\end{figure}

\missingfigure{ROS system used for multiple ships, when doing autonomous
formation control}

\section{Multiple AAUSHIP's}
All of the above design only describes the ROS layout on one ship.
Given that this project is about formation control, that implicitly
means that multiple ships has to operate at the same time, which in
the case of cooperative formation some information has to be shared
between ships. AAUSHIP is equipped with Wi-Fi, which in turn means that
it uses an \ac{IP} based networking scheme. This means that one can
easily add up to 255 ships on one subnet with \ac{IPv4}. If the
formation needs to scale to more than that, some networking design is
needed maybe using \ac{IPv6}.

The degree of information of the information that needs to be share
depends on the formation method chosen. The \ac{ROS} system should
accommodate some arbitrarily defined schemes in a way that is defined
by the control node of the ship. Since every boat is connected with a
via WiFi it is assuming that all ships can reach everyone. This
simplification is used for the testing implementations for now. A more
advanced system, that should take into account that not every ship can
communicate with each other can be developed for large formations or
scalability, but in this project the implementation assumes small
formations with small distance, that is that all ships are reachable
on the whole network. In practice the networks consists of an access
point on any random ship where the others connects to.
