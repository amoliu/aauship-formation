\section{Strategy 1: Potential field formation}
\label{sc:potential-fields}
% Papers of interrest to this topic
A Potential Field Based Approach to Multi-Robot Manipulation \citep{pfmrm}\\
Formation Control with Configuration Space Constraints, \citep{fccsc}\\
UAV Formation Flight using 3D Potential Field, \citep{UAVff3dpf}\\


\subsection{Potentialfield}
In potential fields it is commonly used that a multi-robot group should move in an environment, either with or without obstacles, some papers are \citep{pfmrm}, \citep{fccsc} and \citep{UAVff3dpf}. In the environment there needs to be specified some relative measurements, or states, such that the robots are capable to manoeuvre relative to something. This reference, that the robots should manoeuvre with, is in this case the potential fields. When looking at robots that should keep distance from each other (in a formation), or should move to a desired position (trajectory tracking), the implementation of potential fields come in handy.

The potential field approach can be utilised or implemented in
different ways when taking about formation control. Potential fields
is often used for cooperative control where the formation between
agents is not important, but rather have the goal of moving groups of
agents from one point to another. \todo{Reference papers here} So to
use the potential field mindset it is important to define how these
fields works together to achieve some form of formation that can be
moved after a predefined path. 

Some ideas is to use a individual potential field seen from each
agent and combine this with the other agents local potential field.
Here the potential field has local minima defining the desired
formation. In addition to this there is also a need to define how this
formation can be moved in a predefined path.

Another idea is to use the same potential field, that is a global and
common time varying potential field to define the formation, but this
will require that the agents are already near their formation, such
that every agent will approach their separate local minimum. This can
be used as a leader follower concept where the leader is defining this
global potential field.

These fields can be defined from attraction- and repulsion potential fields, dependent on if the robots needs to move toward or away from a target, described in \citep{pfmrm}, where potential fields are utilized to a virtual structure. When applying a virtual structure the group of robots follow a virtual leader and not an other robot relative to the formation. The potential functions can now be used as forces between the agents to repel them, such that they do not collide, which can be called inter-vehicle forces. The potential functions are also used as forces around the virtual leader such that the agents gets as close as possible to the virtual leader without ever reaching it, which can be called virtual leader forces. This will make a distributed formation around the virtual leader \citep{1655803}. The repelling forces can be utilized in different ways and with advantage used with obstacle avoidance. A great advantage to the virtual leader is that the potential forces can be used to keep the formation and then only generate the trajectory for the virtual leader and not for every individual agent \citep{1655803}.


%A attractive potential can be written in a quadratic way as:
%\begin{align}
%V_{io}^a = \frac{1}{2}k_{io}r_{io}^2
%\end{align}
%\todo{Der skal lige læses i ref [8, 17] for at finde de helt præcise beskrivelser}
%which depends on the position vectors of the robots and the objective, or end position. Likewise can a repulsive potential be expressed as:
%\begin{align}
%V_{ij}^r = k_{ij}/r_{ij}
%\end{align}
%which is dependent on the euclidean distance between the robots.
%The gradient of the potential fields defines the driving force of the robots and the trajectory should be calculated by simulating based on the dynamics %of the system. \todo{Det er vildt svært at finde ud af det her.}

