\section{Strategy 1: Potential field formation}
% Papers of interrest to this topic
A Potential Field Based Approach to Multi-Robot Manipulation,
Formation Control with Configuration Space Constraints,
UAV Formation Flight using 3D Potential Field, 

\subsection{potentialfield}
In potential fields it is commonly used that a multi-robot group should move in an environment, either with or without obstacles.\todo{You say it is common, then we must have multiple references.} In the environment there needs to be specified some relative measurements, or states, such that the robots are capable to manoeuvre relative to something. This reference, that the robots should manoeuvre with is in this case the potential fields. When looking at robots that should keep distance from each other, in a formation, or should move to a desired position, trajectory tracking, the implementation of potential fields come in handy. These fields can be defined from attraction and repulsion potential fields, dependent on if the robots needs to move toward or away from a target, described in \citep{PS:02}. This terminology can be used when a virtual leader is applied to the system. When applying a virtual structure the group of robots follow a virtual leader and not an other robot relative to the formation. The potential functions can now be used as forces between the agents to repel them, such that they do not collide, which can be called inter-vehicle forces. The potential functions are also used as forces around the virtual leader such that the agents gets as close as possible to the virtual leader without ever reaching it, which can be called virtual leader forces. This will make a distributed formation around the virtual leader \citep{1655803}. The repelling forces can be utilized in different ways and with advantage used with obstacle avoidance. A great advantage to the virtual leader is that the potential forces can be used to keep the formation and then only generate the trajectory for the virtual leader and not for every individual agent \citep{1655803}.

%A attractive potential can be written in a quadratic way as:
%\begin{align}
%V_{io}^a = \frac{1}{2}k_{io}r_{io}^2
%\end{align}
%\todo{Der skal lige læses i ref [8, 17] for at finde de helt præcise beskrivelser}
%which depends on the position vectors of the robots and the objective, or end position. Likewise can a repulsive potential be expressed as:
%\begin{align}
%V_{ij}^r = k_{ij}/r_{ij}
%\end{align}
%which is dependent on the euclidean distance between the robots.
%The gradient of the potential fields defines the driving force of the robots and the trajectory should be calculated by simulating based on the dynamics %of the system. \todo{Det er vildt svært at finde ud af det her.}
