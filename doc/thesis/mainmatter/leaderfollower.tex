\section{Strategy 2: Leader-follower}
% Papers of interrest with this topic
Leader-to-Formation Stability,
Cooperative Leader Following in a Distributed Multi-Robot System,
Formation constrained multi-agent control, 

The leader follower principle is used as a higher order principle of how to navigate a group of robots. The principle is that a leader is defined to lead the group of robots in an environment relative to a trajectory. Instead of all the individual robots track their respective trajectories, only the leader follows a trajectory and the following robots keep their position relative to the leader position.

The way that the followers maintain their position can be done in different ways e.g. potential fields \citep{pfmrm} or behavioural methods as Null Space Based behavioural methods \citep{arrichiello2006formation}. With the focus in here the formation should be defined as a 'rigid' formation, where the follower's positions are defined as fixed distances from the leader. The term rigid might not be the best description because the will vary a little all the time, but it is not flexible either \citep{976029}.
The positions should be defined individually for each of the followers to the position of the leader.