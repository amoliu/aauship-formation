\section{Kalman Filter design}
\label{sec:kfdesign}

\subsection{The \acl{KF}}

\nomenclature{$\boldsymbol\Phi$}{State transistion matrix of a discrete linear dynamic system}
\nomenclature{$\mathbf G$}{Input matrix}
\nomenclature{$\mathbf H$}{Measurement sensitivity matrix defining the linear relationship between the state of the dynamic system and the measurements that can be made}
\nomenclature{$\mathbf K$}{Kalman gain matrix}
\nomenclature{$\mathbf u$}{Input vector}
\nomenclature{$\mathbf x$}{State vector of a linear dynamic system}
\nomenclature{$\hat{\mathbf{x}}$}{Estimated state vector}
\nomenclature{$\mathbf{z}$}{Measurement vector}
\nomenclature{$\mathbf{w}$}{Process noise vector}
\nomenclature{$\mathbf{v}$}{Measurement noise vector}
\nomenclature{$\mathbf R$}{Covariance matrix of observational (measurement) uncertainty}
\nomenclature{$\mathbf P$}{Covariance matrix of state estimation uncertainty}
\nomenclature{$\mathbf Q$}{Covariance matrix of process noise in the system state dynamics}


A \ac{KF} is a type of observer that can be applied to estimate the state vector. This is done to filter the measurements from the vessel and smooth these. If the measurements are too noisy, such that the vessel changes direction suddenly, but should be surging forward, a filter can predict the modelled direction and compare this to the measurement.

The \ac{KF} comprises the deterministic part \todo{Deterministic part? nick} of the model which estimates the state vector. This is corrected by means of measurements to estimate the final state vector.

The \ac{KF} is drawn as a block diagram as seen on figure~\vref{fig:blockkf} illustrating both the process and the \ac{KF} together. The upper part represent the process, which can both be a simulated model or the real vessel with measurements, here it is a linear state space model. The lower part is the \ac{KF} which takes in the measurements and estimates the new state vector based on these measurements. The \ac{KF} can be of different types: \ac{LKF}, \ac{EKF} or \ac{UKF}. The figure illustrates the \ac{LKF}. Choosing between these types of filters depends on the type of model used and the application. 


\subsection{\acl{LKF}}
The process model is the usual state space model in discrete form as:
\begin{align}
x_k &= \Phi_{k-1} x_{k-1} + G u_{k-1} + w_{k-1}\\
z_k &= H_k x_k + v_k
\end{align}
\noindent The \ac{LKF} prediction and update can be written as:
\begin{itemize}\tightlist
\item Prediction
\begin{align}
\hat x_k^- &= \Phi_{k-1}\ \hat x_{k-1}^+ + G u_{k-1} \\
P_k^- &= \Phi_{k-1}P_{k-1}^+ \Phi_{k-1}^\top + Q_{k-1}
\end{align}
\item Update
\begin{align}
\bar{\mathbf{z}}_k &= z_k - H_k\ \hat x_k^-\\
S_k &= H_k\ P_k^-H_k^\top + R_k\\
K_k &= P_k^-H_k^\top S_k^{-1}\\
\hat x_k^+ &= x_k^- + K_k \bar{\mathbf{z}}_k\\
P_k^+ &= (I - K_k H_k) P_k^-
\end{align}
\end{itemize}

\begin{figure}
	\centering
	\includesvg{kf_on_sys}
	\caption{Block diagram of a \acl{LKF} resulting in the state estimate $\hat x_k^+$.}
	\label{fig:blockkf}
\end{figure}

Where $P_{k}^-$ is the covariance propagation, $P_{k}^+$ is the update of covariance propagation, $Q$ is a covariance matrix with sensor variances and $R$ is a covariance matrix with model variances. $Q$ is a measure of how much the model is to be trusted. If the variance of the sensors are high this will imply that the model are to be trusted more than the noisy sensor measurements. These variances can sometimes be measured directly at the sensors and used in the $Q$ matrix. This leaves the $R$ matrix as the only design matrix left. $R$ is a measure of how much the measurement are to be trusted. If the variance of the model are high it might be better to trust the actual measurements. $z_k$ is the measurements from the sensors and $\bar{\mathbf{z}}_k$ is the difference between the measurements and the predicted state vector, $\hat x_k^-$. $S_k$ is the covariance matrix of the residual with the variance of the model included. $K_k$ is the optimal Kalman gain, in a \ac{MMSE} sense. 

\subsection{\acl{EKF}}
The above mentioned \ac{LKF} can only be applied on linear systems and transitions. Therefore is this not suited at the AAUSHIP. The position from the \ac{GPS} and the acceleration measurements needs to be rotated with a rotational matrix, which leads to non-linearities in the system. This can be seen on figure \vref{fig:intermediate-calc}. This entails that a \ac{LKF} cannot be used and an \ac{EKF} can be suited. The \ac{EKF} is used to linearise the non-linear terms in the system around the current estimate. In this case it will linearise the transition around the current measurements from the sensors to estimate the true output. The \ac{EKF} can be formulated in discrete form with the prediction and an update as:
\begin{itemize}\tightlist
\item Prediction
\begin{align}
\hat x_k^- &= f(\hat x_{k-1}^-,u_{k-1})\\
P_k^- &= F_{k-1}P_{k-1}^+F_{k-1}^\top+Q_{k-1}
\end{align}
\item Update
\begin{align}
\bar{\mathbf{z}}_k &= z_k - h(\hat x_k^-)\\
S_k &= H_k\ P_k^-H_k^\top + R_k\\
K_k &= P_k^-H_k^\top S_k^{-1}\\
\hat x_k^+ &= x_k^- + K_k \bar{\mathbf{z}}_k\\
P_k^+ &= (I - K_k H_k) P_k^-
\end{align}
\end{itemize}
where the state transition and observation matrices are defined by their respective jacobians:
\begin{align}
F_{k-1} &= \left.\frac{\partial f}{\partial x}\right|_{\hat x_{k-1}^-,u_{k-1}} \label{eq:EKFF}\\
H_k &= \left.\frac{\partial h}{\partial x}\right|_{\hat x_{k}^-}
\end{align}

\subsection{\acl{KF} in the case of AAUSHIP}
\label{sec:kfonaauship}
The input is given by the forces applied to the vessel. On AAUSHIP with the two twin propellers and two side thrusters as illustrated in section~\vref{sec:thrust_allocation}, this will result in forces in the surge and sway direction as well as a torque around yaw, the roll and pitch torques are neglected, such that the input vector $u_k$ becomes.
\begin{align}
u_k = \tau_k =
\begin{bmatrix}
X & Y & 0 & 0 & N
\end{bmatrix}^\top
\end{align}
These are the forces that can be applied by the thrusters mounted at the vessel. It should be noted that the pitch and roll are not set as input, since no thrusters can control these, and should be treated as model variations.

The discrete input matrix $\Gamma$ is a $17 \times 5$ matrix from the discretised $B$ matrix from equation~\vref{eq:ss}. \todo{Should we use $\Gamma$ for $G$ as in done in other sources (e.g. fossen), such that the input matrix is not confused with the restoring force matrix from fossen?}
The $\Gamma$ matrix takes the forces in $X$, $Y$ and $N$ as input, and neglects inputs in $\phi$ and $\theta$. The forces in $\phi$ and $\theta$ are outputs from the system that makes the vessel change in pitch and roll.

The discrete system $\boldsymbol \Phi$ is a $17 \times 17$ from the discretised $A$ matrix from equation~\vref{eq:ss}, but expanded to contain the full state vector calculations.

The dimensions of the different matrices used are checked as a type of sanity check and verify that the matrices are in correct size.

The covariance propagation matrix, the uncertainty of the estimated state, is given by:
\begin{align}
P_k^- &= F_k\ P_{k-1}^-F^\top + Q_k\\
\text{dim}(P_k^-) &= [17 \times 17]\cdot [17 \times 17]\cdot [17 \times 17]^\top + [17 \times 17]
\end{align}
The $Q$ is the variances of each of the states from the full state vector.

The posteriori error covariance matrix, the update, is given by:
\begin{align}
P_k^+ &= (I - K_k\ H_k)P_k^-\\
\text{dim}(P_k^+) &= ([17 \times 17] - [17 \times 7]\cdot [7 \times 17])\cdot [17 \times 17]
\end{align}
which is a measure of the estimated accuracy of the state estimate. Adding a middle calculation as the residual covariance:
\begin{align}
S_k &= H\ P\ H^\top + R\\
\text{dim}(S_k) &= [7 \times 17]\cdot [17 \times 17]\cdot [7 \times 17]^\top + [7 \times 7]
\end{align}
The $R$ is variances from the sensors, which makes it sensor noise terms.

The updated Kalman Gain:
\begin{align}
K_k &= P\ H^\top\ S^{-1}\\
\text{dim}(K_k) &= [17 \times 17]\cdot [7 \times 17]^\top\cdot [7 \times 7]^{-1}
\end{align}
witch is optimal in a \ac{MMSE} sense. %\todo{Hvorfor virkede denne ac ikke? Har fjernet den tidtil videre} - Hah, det var slet ikke den som var fejlen.

State vector
\begin{align}
\hat{\mathbf x}=
\begin{bmatrix}
N & E & x_b & y_b & \phi & \theta & \psi & u & v & p & q & r & \dot u & \dot v & \dot p & \dot q & \dot r
\end{bmatrix}^\top
\end{align}

Measurement vector
\begin{align}
\mathbf z=
\begin{bmatrix}
N & E & \psi & u & v & a_x & a_y
\end{bmatrix}^\top
\end{align}

Measurement matrix defining the relationship between the state of the dynamic system and the measurements
\begin{align}
H_k =
\begin{bmatrix}
1 & 0 & 0 & 0 & 0 & 0 & 0 & 0 & 0 & 0 & 0 & 0 & 0 & 0 & 0 & 0 & 0 \\
0 & 1 & 0 & 0 & 0 & 0 & 0 & 0 & 0 & 0 & 0 & 0 & 0 & 0 & 0 & 0 & 0 \\
0 & 0 & 0 & 0 & 0 & 0 & 1 & 0 & 0 & 0 & 0 & 0 & 0 & 0 & 0 & 0 & 0 \\
0 & 0 & 0 & 0 & 0 & 0 & 0 & 1 & 0 & 0 & 0 & 0 & 0 & 0 & 0 & 0 & 0 \\
0 & 0 & 0 & 0 & 0 & 0 & 0 & 0 & 1 & 0 & 0 & 0 & 0 & 0 & 0 & 0 & 0 \\
0 & 0 & 0 & 0 & 0 & 0 & 0 & 0 & 0 & 0 & 0 & 0 & 0 & 1 & 0 & 0 & 0 \\
0 & 0 & 0 & 0 & 0 & 0 & 0 & 0 & 0 & 0 & 0 & 0 & 0 & 0 & 1 & 0 & 0 
\end{bmatrix}
\end{align}

The covariance matrix of observational (measurement) uncertainty $R_k$ is assumed to be uncorrelated with the other states, such that the matrix only becomes the variances, with no covariance elements as:
\begin{align}
&R_k = \diag{\mathbf{v}} =\\ \nonumber
&\diag{^\text{GPS}\sigma_{N}^2,\ ^\text{GPS}\sigma_{E}^2,\ ^\text{GPS}\sigma_{\psi}^2,\ ^\text{GPS}\sigma_{u}^2,\ ^\text{GPS}\sigma_{v}^2,\ ^\text{IMU}\sigma_{a_x}^2,\ ^\text{IMU}\sigma_{a_y}^2}
\end{align}
These variances can be found by letting the AAUSHIP be in steady state. The sensor outputs are read while in steady state to check the variances from these. This will be the variances of the individual sensor measurements thus the variances of the particular measurements. The test finding these variances can be found in appendix \todo{lav appendix og ref til det her}.

The covariance matrix of process noise in the system state dynamics is $Q_k$ and is assumed to be the variances of each individual state:
\begin{align}
&Q_k = \diag{\mathbf{w}} = \\ \nonumber
&\diag{\sigma_{N}^2, \sigma_{E}^2, \sigma_{x_{b}}^2, \sigma_{y_{b}}^2, \sigma_{\phi}^2, \sigma_{\theta}^2, \sigma_{\psi}^2, \sigma_{u}^2, \sigma_{v}^2, \sigma_{p}^2, \sigma_{q}^2, \sigma_{r}^2, \sigma_{\dot u}^2, \sigma_{\dot v}^2, \sigma_{\dot p}^2, \sigma_{\dot q}^2, \sigma_{\dot r}^2}
\end{align} \todo{bedre notation for process og maalevarians?}
The variances can be found by making tests of the AAUSHIP. These should be carried out such that only one type of variance is tested at a time. By doing this, and testing around a known mean of the given state, makes it possible to test the variance of that particular state. The test finding these variances can be found in appendix \todo{lav appendix og ref til det her}.
\todo{What about $\mathbf w$ and $\mathbf v$? How to determine them correctly}

The $\mathbf F$, equation\ref{eq:EKFF} from the previous calculations, are the $[17 \times 17]$ system matrix, referred to as $\Phi$ in discrete case. This system matrix changes over time due the changes in the heading. The change of heading is not a linear transition which makes the system non-linear and therefore needs to be linearised about the given states of the system. Therefore this matrix is used as the system matrix in the \ac{EKF}, which is implemented in the AAUSHIP. $\Phi$ is designed to be
\begin{align}
\Phi = 
  \left[\begin{array}{c|ccc|c}
    I_{2\times 2} & 0_{2\times 5} & R_z (\psi) & 0_{2\times 3} & 0_{2\times 5}\\ \hline
    0_{10\times2}  & \multicolumn{3}{c}{\multirow{1}{*}{{$A_d$}}} \vline&  0_{10\times 5} \\ \hline
    0_{5\times 7} & & A_{d[6:10]} & & I_{5\times 5}
  \end{array}\right]
\end{align}
\todo{fix lower left corner alignment vline}

\subsection{Determine R}
The matrix R represents covariance matrix of observational (measurement) uncertainty. This is the matrix which sets the individual variances of the specific measurements from the sensors, the measurements from the output vector $z$. This means that the coefficients of the R matrix can be determined within some interval. The disturbances of each sensor measurement highly depends on which sensor is implemented in the used system. When looking at the AAUSHIP the sensors are of higher accuracy, which means that the sensors are to some extend trustworthy. There are two types of sensors in the AAUSHIP, one \ac{GPS} and an \ac{IMU}. There are two types of \ac{GPS}s installed; a \ac{RTK} \ac{GPS} and a standard \ac{GPS}. \todo{Her kan kun naevnes dists from den alm pt - Skriv noget mere efter snak med Nick}. The \ac{IMU} consists of a magnetometer, a gyro and an accelerometer. The magnetometer has a measured accuracy of \todo{Indsaet og forklar hurtigt}. The gyro has a accuracy of \todo{Indsaet for x y z og forklar hurtigt}. The accelerometer has a measured accuracy of \todo{Indsaet for x y z og forklar hurtigt}. The \ac{GPS} have different accuracies dependent on which type is used. The main type used will be the \ac{RTK} \ac{GPS} which has higher accuracy than the standard \ac{GPS}. The resulting R matrix is measured at the measurements of $z$ and determined to be
\begin{align}
&R_k = \diag{\mathbf{v}} =\\ \nonumber
&\diag{^\text{GPS}\sigma_{N}^2,\ ^\text{GPS}\sigma_{E}^2,\ ^\text{GPS}\sigma_{\psi}^2,\ ^\text{GPS}\sigma_{u}^2,\ ^\text{GPS}\sigma_{v}^2,\ ^\text{IMU}\sigma_{a_x}^2,\ ^\text{IMU}\sigma_{a_y}^2}
\end{align}

\subsection{Determine Q}
The matrix Q represents the covariance matrix of process noise in the system state dynamics. This matrix includes parameters as disturbances in the process itself. When looking at the AAUSHIP this could for instance be the incoming waves acting as disturbances both in the roll, pitch and the heading. These types of disturbances can be hard to measure and put a precise number to. Instead this matrix is used as a tuning parameter for the \ac{KF} where this sets a weighting of how much the process, and therefore the model, are to be trusted. If the sensors has a high accuracy, it would be of benefit to trust these more than the process, thus putting the parameter of this specific measurement in the Q matrix high. But the Q matrix is used to combine the measurements and the model prediction together. This means that based on the variances from the Q matrix, based on the values from the R matrix, is tuned to get the best performance from the combination of the model and the measurements.\\
As the model of the AAUSHIP is made makes it possible to tune the parameters of Q in a systematic way. The model decouples the acceleration from the velocity, and the velocity from the position. Therefore it is of benefit to tune the acceleration noise variance firstly, such that this makes a proper fit to a step function. A simple test can be simulated where the AAUSHIP accelerates to a certain velocity and keeps this velocity. This needs to fit within some predetermined margin that makes the AAUSHIP follow a satisfying curve in each of the acceleration, velocity and position. The simulation curve is both based on measurements and model predictions \todo{Vi skal have sat nogle tal på det her, maaske forme en krav spec? - Dog er det jo mest the relative position til trajectory som har en vigtigere krav}. A wanted acceleration curve of the acceleration, for instance in surge, will look like \todo{Få lige sat billede ind}. Afterwards is the velocity of the AAUSHIP tuned. This is done in the same manor, where the velocity of a step function needs to look like \todo{Også billede her}. At last is the position tuned, which is the one with largest relative noise from the measurements. This forces the value in Q to be smaller thus making the AAUSHIP trust the model prediction more than then measurements. A position plot with two different values in Q can be seen here. \todo{lav plot med to forskellige Q værdier i N og E}. The one with the lowest value in Q gives the best position fit of these, which also follows the intuition of how the \ac{KF} should work. The resulting Q matrix is tuned to be
\begin{align}
&Q_k = \diag{\mathbf{w}} = \\ \nonumber
&\diag{\sigma_{N}^2, \sigma_{E}^2, \sigma_{x_{b}}^2, \sigma_{y_{b}}^2, \sigma_{\phi}^2, \sigma_{\theta}^2, \sigma_{\psi}^2, \sigma_{u}^2, \sigma_{v}^2, \sigma_{p}^2, \sigma_{q}^2, \sigma_{r}^2, \sigma_{\dot u}^2, \sigma_{\dot v}^2, \sigma_{\dot p}^2, \sigma_{\dot q}^2, \sigma_{\dot r}^2}
\end{align}

\subsection{When absence of \ac{GPS} signal}
When there is no \ac{GPS} signal it is not possible to make any new estimates based from the measurements. This results in a model based phase of the controlling of the AAUSHIP. Therefore the AAUSHIP needs to converge to the predetermined trajectory based on how the model of the ship would do it. When a new \ac{GPS} signal is present this needs to be taken into account and thereby used as a correction to the model prediction. When the signal is absent there are different ways to handle this in the model. One of the ways is to set the variance of the sensor noise from the \ac{GPS} high. This value could be around $10 \cdot 10^9$ to ensure that the model does not take the measurement from the \ac{GPS} into account.

\subsection{Sample rates}
If the different sensor measurements are not sampled with the same rate will the lower sample rates be untrue. This induces the same situation as if the \ac{GPS} signal is absent, thus setting the variances of these measurements high. The same situation will occur as with the missing \ac{GPS} signal, making the measurement be highly untrusted and only taking in the model prediction for that particular sample.

\subsection{Overall filter test and conclusion}
Ja det er nok meget godt.




%"Det lange af det korte er" - LOL, sagt af en buisness mand i toget, han meste lidt det modsatte. Selvom samtalen faktisk blev rimelig lang :p #FolkFraÅrhus