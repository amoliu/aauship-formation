\chapter{Modelling}
\head{This chapter describes the modelling of AAUSHIP. This is
necessary to be able to use model based control algorithms and
estimators.}

\section{Hydrodynamic Modelling}
Hydrodynamic added mass is defined as the mass added to a system due to an accelerating or decelerating body must move a volume of the surrounding fluid as it moves through it. To this is said that the object and fluid is not able to occupy the same physical space simultaneously.
\begin{align}
M_A \dot \nu_r + C_A(\nu_r)\nu_r + D(\nu_r)\nu_r = \tau
\label{eq:hydmodel}
\end{align}
where
\begin{align}
&M_A \text{is the added mass matrix from the system}\nonumber\\
&C_A \text{ is the added mass matrix due to the Coriolis force}\nonumber\\
&D(\nu) \text{ is both the potential and viscous damping matrices}\nonumber\\
&\tau \text{ is control and propulsion forces}\nonumber\\
&\nu \text{ is the velocities of the vessel in all directions and moments}
\end{align}

\section{Rigid Body Modelling}
The rigid body is used to model the physics of the vessel. It is an idealisation of the solid body from where the physical motions of the vessel are to be derived. Translational motion and rotational motion be derived by analysis of this, and by \citep{fossen} written in component form as:
\begin{align}
f^b_b &= [X,Y,Z]^T & &- \text{force through } o_b \text{ expressed in } \{b\}\\
m^b_b &= [K,M,N]^T & &- \text{moment about } o_b \text{ expressed in } \{b\}\\
v^b_{b/n} &= [u,v,w]^T & &- \text{linear velocity of } o_b \text{ relative } o_n \text{ expressed in } \{b\}\\
\omega^b_{b/n} &= [p,q,r]^T & &- \text{angular velocity of } {b} \text{ relative to } \{n\} \text{ expressed in } \{b\}\\
r^b_g &= [x_g,y_g,z_g]^T & &- \text{vector from } o_b \text{ to CG expressed in } \{b\}
\end{align}

\begin{align}
M_{RB} \dot \nu_r + C_{RB}(\nu_r)\nu_r = \tau_{RB}
\label{eq:rigidmodel}
\end{align}
where
\begin{align}
M_{RB} &\text{ is the system inertia matrix}\nonumber\\
C_{RB} &\text{ is coriolis-centriopedal matrix}\nonumber\\
\tau_{RB} &\text{ is a lumped force combined of } \tau_{\text{hyd}} + \tau_{\text{hs}} + \tau_{\text{wind}} + \tau_{\text{wave}} + \tau\nonumber
\end{align}
where in $\tau_{RB}$
\begin{align}
\qquad \tau_{\text{hyd}} &\text{ is the hydrodynamic force}\nonumber\\
\qquad \tau_{\text{hs}} &\text{ is the hydrostatic force}\nonumber\\
\qquad \tau_{\text{wind}} &\text{ is the wind force}\nonumber\\
\qquad \tau_{\text{wave}} &\text{ is the wave force}\nonumber\\
\qquad \tau &\text{ is the control and propulsion forces}\nonumber
\end{align}

\section{Total Model of Vessel}
\begin{align}
\underbrace{M_{RB} \dot \nu_r + C_{RB}(\nu_r)\nu_r}_{\text{rigid-body forces}} + \underbrace{M_A \dot \nu_r + C_A(\nu_r)\nu_r + D(\nu_r)\nu_r}_{\text{hydrodynamic forces}}  = \tau + \tau_{RB}
\label{eq:totmodel}
\end{align}
\todo{Argumenter via reference til f.eks. fossen at vi kan se bort fra
nogle led. Er det en valid antagelse?}
Since the vessel within this project is of smaller scale, the $M_A$, $C_A$ and $C_{RB}$ from \ref{eq:hydmodel} and \ref{eq:rigidmodel} are neglected. $M_A$ is the added mass and is as a start omitted due to the tests needs to be made as an object moving through the water with some drag. If the model needs to be further improved in the process this is a place to start modelling. The coefficients of $M_A$ are rather inconvenient to determine without advanced equipment like a towing tank, where constant velocity can be applied and measure drag and more in all directions and moments. $C_A$ and $C_{RB}$ represents forces due to a rotation of the body frame, \{$b$\}, about the inertial frame, the NED frame. These are omitted as well due to the small vessel where the body frame is placed in a predefined local frame which acts as the NED frame. This reduces equation \ref{eq:totmodel} down to the following
\begin{align}
M_{RB} \dot \nu_r + D(\nu_r)\nu_r = \tau_{RB} + \tau
\label{eq:reducedmodel}
\end{align}
The damping matrix which contains the coefficients of the drag is denoted the hydrodynamic damping matrix. This consists both of $D$ which is the linear damping matrix due to potential damping and possible skin friction with the water and $D_n(\nu_r)$ which is the nonlinear damping matrix due to quadratic damping and higher order terms. This can be expressed as $D(\nu_r) = D + D_n(\nu_r)$ This will, as a start be modelled as the linear part, being potential and viscous damping. As higher velocities will the nonlinear part become more dominant due to the quadratic terms of the velocity, thus is mostly used at faster vessels. The linear damping matrix $D$ contributes more at lower speed manoeuvring and stationkeeping. Therefore is the damping matrix $D$ used, and is expressed by ~\citep{fossen} for a 6 \ac{DOF} system to be
\begin{align}
D =-
\begin{bmatrix}
X_u & 0 & 0 & 0 & 0 & 0\\
0 & Y_v & 0 & Y_p & 0 & Y_r\\
0 & 0 & Z_w & 0 & Z_q & 0\\
0 & K_v & 0 & K_p & 0 & K_r\\
0 & 0 & M_w & 0 & M_q & 0\\
0 & N_v & 0 & N_p & 0 & N_r
\end{bmatrix}
\end{align}

The rigid-body system matrix of the vessel is given for a 6 \ac{DOF} system by ~\citep{fossen} as:
\begin{align}
M_{RB} &=
\begin{bmatrix}
m\boldsymbol{I}_{3x3} & -m\boldsymbol{S}(r^b_g)\\
-m\boldsymbol{S}(r^b_g) & \boldsymbol{I}_b
\end{bmatrix}
\nonumber\\
&=
\begin{bmatrix}
m & 0 & 0 & 0 & mz_g & -my_g\\
0 & m & 0 & -mz_g & 0 & mx_g\\
0 & 0 & m & my_g & -mx_g & 0\\
0 & -mz_g & my_g & I_x & -I_{xy} & -I_{xz}\\
mz_g & 0 & -mx_g & -I_{yx} & I_y & -I_{yz}\\
-my_g & mx_g & 0 & -I_{zx} & -I_{zy} & I_z
\end{bmatrix}
\end{align}

This will be reduced to a 5 \ac{DOF} system due to the fact that the
vessels buoyancy cannot be controlled as such. The vessel will always
be on the water surface and this removes the degree of freedom which
is the heave, the change of $z$ position of the vessel. A 5 \ac{DOF}
system will be modelled as;
\begin{align}
M_{RB} =
\begin{bmatrix}
m & 0 & 0 & mz_g & -my_g\\
0 & m & -mz_g & 0 & mx_g\\
0 & -mz_g & I_x & -I_{xy} & -I_{xz}\\
mz_g & 0 & -I_{yx} & I_y & -I_{yz}\\
-my_g & mx_g & -I_{zx} & -I_{zy} & I_z
\end{bmatrix}
\end{align}
and
\begin{align}
D = -
\begin{bmatrix}
X_u & 0 & 0 & 0 & 0\\
0 & Y_v & Y_p & 0 & Y_r\\
0 & K_v & K_p & 0 & K_r\\
0 & 0 & 0 & M_q & 0\\
0 & N_v & N_p & 0 & N_r
\end{bmatrix}
\label{eq:damping-matrix}
\end{align}
where the heave are neglected from the 6 \ac{DOF} system. In the principle could a 3 \ac{DOF} system be enough to make the control to the vessel and make it manoeuvre in the water, but as the scope is to exploit the sonar to map the seabed it would be beneficial to implement the roll and pitch as well and make the system as a 5 \ac{DOF}.

\section{Identification of Hydrodynamic Derivatives}
\label{sec:hydrocoeff}
The linear kinetic model \eqref{eq:reducedmodel1}, which consists of the mass matrix $M_{RB}$ and the damping matrix $D$.
\begin{align}
M_{RB} \dot \nu_r + D\nu_r = \tau_{RB} + \tau
\label{eq:reducedmodel1}
\end{align}
The coefficients of the model needs to be determined before the model can be simulated and implemented. These coefficients can be determined in multiple ways. Often ship design companies are able to use \ac{CFD} to determine the coefficients, or make use of a towing tank to determine the coefficients. These applications are often expensive and proprietary. So a third method to do this is to perform some simple tests to do approximations of the coefficients. To do so some assumptions needs to be made. The model looks like:
\begin{align}
M_{RB} \dot \nu_r + D\nu_r = \tau_{hyd} + \tau_{hs} + \tau_{wind} + \tau_{wave} + \tau
\end{align}
Since the tests will be pjackerformed in still water some of the forces can be neglected. The $\tau$ is the only force to be taken into account to perform the tests. This is the input to the vessel. The delimitation can be assumed due to the forthcoming procedure of the tests.

Two different types of measurements needs to be made. These are dependent on which coefficients that is wanted. The first type of test is split into three phases, as seen on figure \ref{fig:acceldecel}, and is approximated from a first order fitting. The second type of test is made as seen on figure \ref{fig:harmonic-damping}, and is approximated from a second order fitting.

In the first type of test is the vessel accelerated to constant velocity. When the constant velocity is ensured, the input force to the vessel is removed and zero input is therefore applied. This will correspond to a model like:
\begin{align}
M_{RB} \dot \nu_r + D\nu_r = 0
\label{eq:decelmodel}
\end{align}
An acceleration, constant velocity and deceleration will look like figure \ref{fig:acceldecel}.
\begin{figure}[htbp]
	\centering
	\includesvg[width=0.5\textwidth]{acceldecel}
	\caption{An acceleration followed by constant velocity followed by zero input.}
	\label{fig:acceldecel}
\end{figure}
\begin{figure}[htbp]
	\centering
	\includesvg[width=0.5\textwidth]{harmonic_damping}
	\caption{Pitch and roll response.}
	\label{fig:harmonic-damping}
\end{figure}
This makes it possible to determine some of the coefficients of the $D$ matrix. e.g. is the damping in the x direction determined by:
\begin{align} 
M_{11} \ddot x + D_{11} \dot x = 0
\label{eq:noinput}
\end{align}
The mass of the vessel can be measured, being the $M_{11}$. The velocity and acceleration can be estimated from measurements of the positions measured by the \ac{GPS} at the vessel. This makes the damping coefficient $D_{11}$ the only unknown in equation \ref{eq:noinput}. From this a linearisation can be made, see appendix \ref{app:damping}. From this can the damping coefficient be determined.

This makes it possible to determine the input force by applying a step input on the motors and let the vessel accelerate to the same constant velocity. This can be done since, due to the previous test, only the input is unknown. A model of this will look like:
\begin{align} 
M_{11} \ddot x + D_{11} \dot x = \tau
\label{eq:maxinput}
\end{align}
From this it is possible to estimate the input force as a linearisation, since this is the only unknown from equation \ref{eq:maxinput}.

This type of procedure is used in all of the first type tests and can be put into steps.
\begin{enumerate}
	\item Step one is to apply force in one direction until a constant velocity is achieved and then measure the damping while the vessel is decelerating to estimate the damping coefficient.
	\item Step two can be made after knowing the damping coefficient. Then is only the input unknown and a step can be applied to accelerate the vessel to constant velocity again. From this step input can a input force be estimated.
\end{enumerate}

The second type of test is used when determining coefficients as pitch and roll, $r$ and $p$. In these tests is the vessel put to a maximum angle in pitch or roll and is afterwards released. This will make the vessel converge to steady state, as seen on figure \ref{fig:harmonic-damping}. This is due to the acting restoring force due to that the vessel is perturbed away from its equilibrium and converges back to it. The convergence is dependent on the angle of either pitch or roll of the vessel, and therefore the tests need to be made from the same angle every time.

When determining the forces in sway due to rolling, $Y_p$, the following equation needs to be outlined:
\begin{align}
M_{RB} \dot \nu + D \nu = \tau
\end{align}
which for roll will be:
\begin{align}
M_{RB2} \dot \nu =
\begin{bmatrix}
0 & m & -mz_g & 0 & mx_g
\end{bmatrix}
\dot \nu
\label{eq:rollmedmere1}
\end{align}
and
\begin{align}
D_2 \nu =
\begin{bmatrix}
0 & Y_v & Y_p & 0 & Y_r
\end{bmatrix}
\nu
\label{eq:rollmedmere2}
\end{align}
Since the roll is uncontrollable should the equation equal zero, due to no input. When taking the parts from equation \ref{eq:rollmedmere1} and \ref{eq:rollmedmere2} that consider only roll, and with no input, it reduces to the following equation:
\begin{align}
-mz_g \dot p + Y_pp = 0
\end{align}
This can be approximated to a second order equation fulfilling the convergence as shown on figure \ref{fig:harmonic-damping}. From this it is possible to determine the damping in roll due to a acting force in sway, $Y_r$.

One of the important things are to decouple the tests such that the damping coefficients can be measured. This can, as a start, be performed in x-, y- and z-directions. The mixed damping coefficients, as the $Y_r$ (the force in y-direction due to a rotation), has got many components as shown above. But, after have performed the previous tests, these will become the next unknowns to be determined. Looking at the system as a 5 \ac{DOF} will make it possible to determine the coefficients needed. Some of the coefficients can be found from measurements made by \todo{ref til Rasmus og Brian}. The decoupled coefficients are tested from the equations in \ref{eq:syseq}.
\begin{subequations}
\begin{align}
M_{11} \ddot x + D_{11} \dot x = \tau\\
M_{22} \ddot y + M_{23} \ddot \psi + D_{22} \dot y + D_{23} \dot \psi = \tau\\
M_{45} \ddot y + M_{55} \ddot \psi + D_{45} \dot y + D_{55} \dot \psi = \tau
\end{align}
\end{subequations}
being
\begin{subequations}
\begin{align}
m \ddot x - X_u \dot x = \tau\\
m \ddot y + mx_g\ddot\psi - Y_v \dot y - Y_r \dot \psi = \tau\\
mx_g \ddot y + I_z\ddot \psi - N_v \dot y - N_r \dot \psi = \tau
\end{align}
\label{eq:syseq}
\end{subequations}
While testing, as seen in appendix \ref{app:damping}, the vessel will in one test surge straight forward, in the next only perform sway and in the last only perform a rotation. This will, in the second test, make the terms regarding rotation be zero. Therefore it becomes possible to determine the remaining coefficients. This is also done in the last test while rotating. By making the vessel be in the same spot, then any movement in y can be neglected. This makes it possible to determine the damping while rotating. After these three tests it is possible to determine the last cross terms from the system. The equations to determine will look like:
\begin{align}
m \ddot x - X_u \dot x = \tau\\
m \ddot y - Y_v \dot y = \tau\\
I_z\ddot \psi - N_r \dot \psi = \tau
\end{align}
After determine the damping coefficients $X_u$, $Y_v$ and $N_r$ it is possible to determine the last parameters from the original system from \ref{eq:syseq}, and can be seen here:
\begin{align}
m \ddot y + mx_g\ddot\psi - Y_v \dot y - Y_r \dot \psi = \tau\\
mx_g \ddot y + I_z\ddot \psi - N_v \dot y - N_r \dot \psi = \tau
\end{align}
$Y_r$ and $N_v$ are the only unknowns and the rest to be determined. The performed tests to measure these coefficients can be found in appendix \ref{app:damping}.
