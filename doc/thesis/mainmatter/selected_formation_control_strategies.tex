\chapter{Selected Formation Control Strategies}
\label{ch:selformctrl}
\head{This chapter investigates the previously mentioned formation control aspects from section \ref{ch:formcontrol} with focus in three different types of formation control. These three types are analysed with focus on the specific task applied at the Port of Aalborg, afterwards simulated to be able to conclude which of these are the most beneficial to implement at the AAUSHIP fleet.}

\section{Relevant Characteristics}
To have a tool to determine what strategy is the best suited for the
problem described in \vref{sc:mission}, some parameters that describes
different characteristics of a strategy is needed. Listed here is a
description of each key parameter, also mentioning what is desired for
the mission at hand. The characteristics are weighted on a scale from $-3$ to $3$ to be able to pick out the relevant formation as a conclusion.

\begin{description}
\item[Communication] The communication requirements should be a measure of how much bandwidth is used. If the communication bandwidth usage is low it should be rated as a good thing, since the low bandwidth implies that the communication can be performed more easy and the scalability will become less complicated. It is not of importance if the communication breaks a link of the agents, this will be discussed in the individual formation control categories. If the communication has a score of $3$, this implies that the bandwidth usage is low and if the score is $-3$ means that the bandwidth usage is high. If only one way communication exists this will probably also lead to lower communication, thus also making a higher score.

\item[Control architecture] The control architecture is a weighting of the complexity of the control structure. A high weighting will mean that the complexity is low and thus less complicated to implement. It will also be combined with a relation to the type of controller, where a more simple controller can be better if the overall objective can be accomplished. If the control can be done with a less complicated controller, i.e. a \ac{LOS} controller instead of some a more complex controller like \ac{MPC}, this can be preferred. The amount of needed inputs to the controller will also be a weighting, thus if the numbers of inputs are high this will be weighted low. The weighting of the control structure is a combination of the above mentioned, where a complex controller with high number of inputs will be weighted as $-3$ and a simple controller with low input will be weighted $3$.

\item[Obstacle avoidance] The obstacle avoidance does not have a specific rating in the use case of the AAUSHIP due to the assumption of a open water manoeuvring. This means that the weighting of this should be taken as neutral and the implementation will be further work. Yet it is described in the specific control strategies because the implementation of this can be useful in the further work of the AAUSHIP projects. The obstacles can be known pre mission or they can be detected during mission and avoided and the task of the avoidance will take different shapes dependent of the specific task. The weighting will be made such that if the obstacle avoidance is easy to implement, it will be weighted $3$, and if it is not present it will be weighted $0$. The reason why no obstacle avoidance is weighted as $0$ is because the criteria does not include obstacle avoidance, but the further work has a weighting to implement it.

\item[Transients] The transients will be a weighting of how well the formation can make a turn and meanwhile keep the formation relative to the path. If the formation are unable to track the path during a turn it will be weighted low ($-3$) and will be the primary weighting. If the formation are able to track the path, but the formation will deviate a little, this will also be a down rating but not as much as if it is unable to track the path. This will i.e. get a weighting of $1$. If the formation tracks the path perfectly and the formation is kept rigid through the transient the weighting will be $3$. Some formation problems during transients can be solved by the generation of the path but this will result as a limitation to the path generation. 

\item[Scalability] The weighting of the scaling is done on two criteria. It is a combination of the structure of the complete formation and the control architecture and an estimate of the bandwidth usage from the higher need of communication. This weighting will be a summed weighting, thus both bandwidth and formation complexity is weighted equally. A rather simple expansion of the formation structure will be preferable, and a relative low bandwidth usage is good. If only communication from a single leader to one respective follower is required, this will be a good thing because it leads to half the bandwidth usage. If the bandwidth usage is high and the control architecture is complex this will lead to harder scalability, which in that case will be weighted $-3$. If the control architecture is low and the input to the controller is low, the bandwidth is low and the communication layer is simple this leads to simple scalability which will be weighted $3$.

\item[Failure]
The failure criteria in these topics are only related to the output of the failure. It is not of relevance if the failure arises in sensors, in actuators or in the control structure, but only what will happen to the mission if one agents fails out. The degree of what will happen will choose the weighting such that if one agent fails and every other agent also fails the mission, this will be weighted as $-3$. If a single agent fails and nothing else happens will lead to a weighting as $3$, because if an agent will fail and the mission still can be completed it is preferable.
\end{description}

There are also some parameters that is hard to tell by the principle
of the method, which is listed below. These can be used to compare the
strategies after implementation.
\begin{description}
\item[Preparation time] This is related to the mission setup time. How
	much manual labour is needed to prepare the agents for a mission?
	This could for instance be trajectory generation and how the
	specific pattern needs to be generated. If the formation control strategy is chosen such that the agents needs to be close to the specific formation this will also take preparation time to place them relative to the desired formation.
\item[Time] The mission time for covering the area in a boustrophedon
	pattern. This will be from the start of the mission to a complete mission.
\item[Energy efficiency] This is a theoretical measure of the
	formation efficiency. The energy efficiency of any autonomous systems is relevant because it limits how long the autonomous system can operate autonomously, given that it cannot easily recharge or is fatal for the mission. This means that optimising the energy efficiency is desirable.
\item[Wear and tear] Is the controller aggressive? Aggressive
	controllers is known to make more wear and tear on the actuators. So
	if the strategy can be chosen to minimize the wear and tear this
	could be of benefit to the hardware.
\end{description}

%% Oversigt af control stats

\section{Methods}

\subsection{Direct LOS guidance}
The \ac{LOS} guidance makes the basis for a formation where one leader has a follower connected and this follower has another follower connected and so on. By having this form of guidance the leader will always control the formation and how the development of the formation should be. The leader does not need to be a specific agent but can be a virtual leader, a point which is seen as leader to all the respective followers. The reference is to the leader can be a guidance reference where the leader needs to track a determined path. The reference to the followers can then either be given as a position offset, a distance and angle offset or only a distance to maintain and then make a direct pursuit of the leader.
\subsubsection{Duckling formation}
\label{sc:duckling}
This strategy takes the rise in a duckling or snake formation. In principle is this a leader that has a follower that has a follower and so on. Thereby will the formation take shape as a snake, when the leader takes off and the followers keep the line of sight to their respective leaders. The communication needed in the case of followers only follow one leader, and the communication only works in one direction, then the bandwidth usage is minimal. This implies that the communication it low and simple which is preferable. Similar to the communication will the control architecture become of the simple type where each agent only need to have a \ac{LOS} controller, such that they track their respective leaders. As such is obstacle avoidance not a native implementation of the formation, but can be implemented for the main leader of the formation. By doing this arises a problem of transients where the followers will 'cut corners'. They will directly pursuit their respective leaders, such that the formation will take sharper corners than the main leader. Adding to that if one agent fails during mission every agent after this in the chain fails. Although the formation does not perform well it has the advantage that the scalability is simple. When linking the formation in the duckling formation a new follower can be attached at the last agent and a new follower to this agent and so on. This implies that, in theory, an infinite tail of followers can be attached.

\subsubsection{Echelon formation}
The echelon formation is a branch from the duckling formation. In this formation the followers to the leader is not in a tail of the leader, but is offset into a echelon formation. The principle being the simple \ac{LOS} controller and the communication as followers to their respective leaders still apply in this branch, but the reference is given to the followers as a offset with an angle and a degree. This implies that the problem statements from section \ref{sc:duckling} also applies to the echelon formation, and the main difference is the shape of the formation and how the reference is passed on.

\subsubsection{\ac{FRP} with path}
This method only makes use of one path for the whole formation. The agents in the setup will be defined with relative positions to a \ac{FRP} at each time step, such that the formation stays rigid and the arms from the \ac{FRP} to the agents are constants. This setup will in principle work as the followers working in a rigid formation where only the point on the path (the \ac{FRP}) determines where the agents need to go. The reference given to the agents will be a position in the \ac{FRF}, and the formation will have the origo (turning point) in the \ac{FRP}. The individual agents will not be able to communicate, due to the rigid formation, but only keep the position relative to the \ac{FRP}. If the formation stays rigid during transients there will be no problem with collision. If one single agent in the \ac{FRF} fails it will just fall out of the formation and the rest will continue the mission.

\subsection{Precomputed individual paths}
\subsubsection{Full communication}
The implementation of this strategy needs to know for each time step for each agent when the goal at the trajectory is. By doing this it is possible to make the agents be in formation at all times because they have individual positions at their respective paths to specific times. If one of the agents does not reach its position in time, the rest of the agents should stop, or slow down, such that the missing agent has the possibility to catch up. This will change the 'time goals' for the rest of the agents, but the agents will keep the formation. The other option is that the slower agent speeds up, if the agents are not already working at full speed. In this formation principle every agent needs formation about where the others are, which might not be preferable when looking at the bandwidth usage and the scalability of this principle. The same problem will arise when looking at the control architecture, which will only expand more and more when adding more agents. If every agent needs to know information for the others, and act dependent on every single other agent, then the decision and the task allocation will become more complex. When looking into the problem of transients and 'cutting corners', the strategies with individual paths does not have any problems with this. If the formation does cut corners and fail in transients, it is mainly a design error of the path, such that the paths intersect each other. If one of the agents fail during mission then rest will keep close to that agent, unless some kind of exception handling has been made, such that the rest can continue the mission.

\subsubsection{Limited communication}
The overall principle of this implementation is the same as when full communication, the one mentioned above. The difference is that the communication will be greatly decreased when looking at the bandwidth usage and how many agents that needs to know of each other. In this formation strategy agent $i$ only knows information about the neighbour agents, $i_{-1}$ and $i_{1}$.By doing this the information about a slower, or even failed, agent needs to propagate through the formation. If the formation is relatively small, it will not make any remarkable change in the formation structure. If the formation is of a larger size, then the formation will not be as rigid as the one with full communication. If the outer most agent for some reason slows down, then the neighbour needs to act according to that, and the the neighbour of the neighbour needs to act according to that and so on. This will make the formation, in this situation, become skewed, but will go into formation again over time. This is a downside when looking at the transients of the formation and how rigid it is. If the given agent completely fails the mission and breaks down, then the same applies as if there were full communication. An exception needs to be programmed such that the formation will leave the failed agent. The advantage of this strategy is that the control architecture and the communication level will become less complicated, and the scalability to in theory infinitely many agents becomes possible.

\subsubsection{No communication}
Again is the main principle the same for this strategy, only that no information is shared between the agents. If no information is shared between the agents it is assumed that they do not experience any faults. If one of the agents has a fault, that it changes course, slows down or fails in the mission, it will make the formation to fail. The individual agents will still follow their respective paths, and therefore reach their end goals, but not at the same time. If the agent has failed the mission the others does not take this into account, but instead continues their individual mission. The communication in this strategy is very low, since the agents does not exchange information at all. The control architecture also becomes less complex while this strategy only controls the individual agents i.e. with a simple \ac{LOS} controller to track their individual paths. If the individual paths have been generated properly there should not be any problems relating the transients, if assumed that every agent follows its path with acceptance. If it does not, and deviate a little, the formation is not rigid. If one agent fails in the mission, it will also lead to formation deviation, but the rest of the agents will keep their respective goals. Though when working as individual agents, this also implies that the scalability becomes less complex because the control is applied at the individual agent that follows its own path.

\subsection{Potential field}
\subsubsection{Full communication}
In the strategy with potential fields the level of communication can also vary. If the communication can be shared by all the agents within the potential field, the communication is said to be full. When this is the case, every agent has the information about every other agents position and potential field. This makes the formation, made by the potential field, able to correct if any agent starts to fall behind. This concept is the same as with the precomputed paths, where the communication also could be full. In this case the bandwidth usage is also larger when having full communication than with lower communication. Therefore is the communication level also more complex then every agent needs to exchange information with each other. When every single agent need to be dependent on all the other agents, the constraint level gets higher and the complexity of the needed control architecture also becomes higher. When every agent knows the potential fields of the others there should not become any collision problems in transients. But the formation in general is not as rigid when working with potential fields, because of the principle of attraction and repelling. If one single agent fails, breaks down, the others should still know of its appearance such that they can use the potential field to manoeuvre away from the agent. This is mainly to avoid collision. When the communication and the control architecture becomes more complex under full communication, the scalability also becomes more complex. It is not impossible to scale, but a scaling in theory to infinity will become very complex.

\subsubsection{Limited communication}
When the communication is said to be limited it means that not all agents can communicate with each other. The communication range of the agents are reduced, such that i.e. agent $i$ only can communicate with agent $i_{-1}$ and $i_{1}$, but the range can also be large, $i_{-r}$ and $i_{r}$, where $r$ is the communication radius. When doing this the same thing will happen as in the case of limited communication with the individual paths. If one agent starts falling behind then the nearby agents starts to reach first, which then leads to that all agents will reach on the slower agent. Dependent on the formation this will make different hurdles. The formation can try to manoeuvre dependent on the slower agent, but they meanwhile have to follow the potential field map, the given trajectory. In the case of limited communication the same things apply to transients as when full communication, but the formation might diverge more from the determined due to the passing of information between the agents. If one agent breaks down the same goes as with full information, because only the closest agents needs the information about the breakdown to avoid collision. The rest can still continue the mission afterwards. The scaling of the formation becomes a little less complicated than if the agents needed to know information from every agent. Though the potential field mapping will still be complex, and the trajectory generation can be hard.

\subsubsection{No communication}
In the case where no communication are to find in the system the agents will act like no potential field exists from the other agents, and try to follow the potential field trajectory. This will make every agent, independent of the others, try to follow the trajectory and thereby be in the same point. This will over time make the agents collide, if not on the way to the mission goal then at the mission goal. This means that in this case they need a individual potential field trajectory. When making individual potential field trajectories will the scenario be as in the case of individual paths, where the agents will be controlled individually but with respect of a potential field trajectory. This may also lead to colliding agents but it is not necessary as it will be if they followed the same trajectory. If they follow the same reference trajectory a failed agent will result in all the others following behind that will collide resulting in a complete mission failure. Though if the agents instead have their own respective reference trajectory one failing will not fail the mission. If the formation are unlucky, they can collide, but not necessarily.

\begin{sidewaystable}
\begin{tabular}{l|lll|l}
\toprule
\textbf{Approach} & \textbf{Communication} & \textbf{Control architecture} & \textbf{Obstacle avoidance} & ---\\
\hline
\textbf{Leader follower:}&&&& \\
Duckling & Simple, 3 & Simple LOS, 3 & None, 0 & \\
Echelon & Simple, 3 & Simple LOS, 3 & None, 0 & \\
\ac{FRP} with path& Low communication, 0 & Simple, 2 & None, 0 & \\
\textbf{Precomputed individual paths:}&&&& \\
Full communication& High bw, 0 & Multi agent info, 0 & None, 0 & \\
Limited communication& Middle bw, 1 & Local agent info, 1 & None, 0 & \\
No communication& Low bw, 2 & Single agent info, 2 & None, 0 & \\
\textbf{Potential field:}&&&& \\
Full communication& High bw, 0 & Multi agent info, 0 & Natural, 3 & \\
Limited communication& Middle bw, 1 & Local agent info, 1 & Natural, 3 & \\
No communication& Low bw, 2 & Single agent info, 2 & Natural, 2& \\
\bottomrule
 & \textbf{Transients} & \textbf{Scalability} & \textbf{Failure}& \textbf{Sum}\\
\hline
\textbf{Leader follower:}&&&& \\
Duckling  & improper path, -1 & Easy duplicate, 3 & Formation continue, 2 & 10\\
Echelon  & Improper path, 0 & Easy duplicate, 3 & Formation continue, 2 & 11\\
\ac{FRP} with path   & Improper path, 0 & Duplicable, 2 & Collapse, -2 & 2\\
\textbf{Precomputed individual paths:}&&&& \\
Full communication   & No risk, 3 & More bandwidth, -1 & Low failure, 2 & 4\\
Limited communication  & Risk, 2 & Possible, 1 & Can occur, 0 & 5\\
No communication   & Risk, 1 & Individual paths, 3 & High risk, -2 & 6\\
\textbf{Potential field:}&&&& \\
Full communication  & No risk, 3 & More bandwidth, 0 & Low failure, 2 & 8\\
Limited communication  & Risk, 1 & Possible, 1 & Low failure, 2 & 9\\
No communication  & High risk, -2 & Single fields, 3 & Risk, -1 & 6\\
\bottomrule
\end{tabular}
\caption{Decision matrix for the formation strategies. The strategy row from the upper half of the table continues on the lower half, with the points summed in the last column in the lower half.}
\label{tab:decision-matrix}
\end{sidewaystable}

The sum is based on the above mentioned categories and is a weighting from $-3$ to $3$ including $0$. The focus will be placed on the three strategies that scores the highest rating. These can be seen in table \ref{tab:decision-matrix} to be \textbf{Duckling formation}, \textbf{Echelon formation} and \textbf{Potential field formation}. The duckling- and echelon formations have achieved the high rating due to their simplicity and their relatively easy implementation possibilities, while the potential field formation is of higher complexity but still fulfils the criteria from the analysis. When analysing these three strategies it becomes clear that the difference between duckling formation and echelon formation regarding implementation is not very different. The duckling formation will become a direct intercept from follower to leader, where in theory an infinite number of agents can be attached. These all need to fulfil a distance reference in between to avoid colliding, and a trajectory will need to take wide turns to ensure that the absolute speed of every agent stays above a threshold. If the speed decreases it will lead to the agent over steering thus needs to over actuate in the opposite direction. This will lead to oscillations in the trajectory tracking and a risk of colliding becomes relevant. The echelon formation is an offset between a follower to the leader, where the same applies as for the duckling formation. It is also possible to expand this to an infinite number of agents that couples the formation together. There the agents also need to fulfil a distance reference to withhold the determined distances in between and thus also a speed reference. The last formation is the potential field, which becomes more abstract to interpret. The theory of setting control structures for the potential field is designed of attracting and repulsing forces and be using those also implements obstacle avoidance. The formation strategies are explained in more detain in the following.

%% Approaches
\section{Strategy 1: Potential Field Formation}
\label{sc:potential-fields}
% Papers of interrest to this topic
% A Potential Field Based Approach to Multi-Robot Manipulation \citep{pfmrm}\\
% Formation Control with Configuration Space Constraints, \citep{fccsc}\\
% UAV Formation Flight using 3D Potential Field, \citep{UAVff3dpf}\\


\subsection{Potentialfield}
In potential fields it is commonly used that a multi-robot group should move in an environment, either with or without obstacles, some papers are \citep{pfmrm}, \citep{fccsc} and \citep{UAVff3dpf}. In the environment there need to be specified some relative measurements, or states, such that the robots are capable to manoeuvre relative to something. This reference, that the robots should manoeuvre with, is in this case the potential fields. When looking at robots that should keep distance from each other (in a formation), or should move to a desired position (trajectory tracking), the implementation of potential fields come in handy.

The potential field approach can be utilised or implemented in
different ways when taking about formation control. Potential fields
is often used for cooperative control where the formation between
agents is not as important, but rather have the goal of moving agetns into formation and then make groups of
agents move from one point to another \citep{5504176}. So to
use the potential field mindset it is important to define how these
fields works together to achieve some form of formation that can be
moved after a predefined path. 

Some ideas is to use an individual potential field seen from each
agent and combine this with the other agent's local potential field.
Here the potential field has local minima defining the desired
formation. In addition to this there is also a need to define how this
formation can be moved in a predefined path.

Another idea is to use the same potential field, that is a global and
common time varying potential field to define the formation, but this
will require that the agents are already near their formation, such
that every agent will approach their separate local minimum. This can
be used as a leader follower concept where the leader is defining this
global potential field.

The potential fields can be defined from attraction- and repulsion forces, dependent on if the robots need to move toward or away from a target, described in \citep{pfmrm}, where potential fields are utilized with a virtual structure. When applying a virtual structure the group of robots follow a virtual leader and not another robot relative to the formation. The potential functions can now be used as forces between the agents to repel them, such that they do not collide, which can be called inter-vehicle forces. The potential functions are also used as forces around the virtual leader such that the agents gets as close as possible to the virtual leader without ever reaching it, which can be called virtual leader forces. This will make a defined formation around the virtual leader \citep{1655803}. The repelling forces can be utilized in different ways and with advantage used with obstacle avoidance, where these need to contribute with a repelling force to the potential field, making agents diverge from these objects. A great advantage to the virtual leader is that the potential forces can be used to keep the formation and then only generate the trajectory for the virtual leader and not for every individual agent \citep{1655803}. The virtual can also be an agent as the reference point in the formation but is mostly thought of as the reference point for all agents in the formation.



\section{Strategy 2: Leader-Follower}

The leader follower principle is used as a higher order principle of how to navigate a group of robots. The principle is that a leader is defined to lead the group of robots in an environment relative to a trajectory. Instead of all the individual robots track their respective trajectories, only the leader follows a trajectory and the following robots keep their position relative to the leader position.

The way that the followers maintain their position can be done in different ways e.g. potential fields \citep{pfmrm}, behavioural methods as Null Space Based behavioural methods \citep{arrichiello2006formation} or versions of a direct \ac{LOS} algorithm. With the focus in here the formation should be defined as a 'rigid' formation, where the follower's positions are defined as fixed distances from the leader. The term rigid is not a strict description in this case because the formation can vary a little all the time, but it is not flexible either \citep{976029}.

The positions should be defined individually for each of the followers to the position of the leader. This can be done through a formation constraint function, F($\eta$), which should be a strict convex function to ensure the formation constraint (the actual formation). This formation constraint function can be expressed in different ways, with the actual positions of the agents and the virtual leader or positions of the agents relative to the virtual leaders starting position.

From the decision table \ref{tab:decision-matrix} it can be seen that duckling formation and echelon formation have got a high rating. These are both branches of the leader-follower principle which often shows to be applicable in formation issues \citep{TKP04}, \citep{1013687}, \citep{976029}.

The duckling formation is a direct intercept algorithm, where the followers intercept their respective leaders and withholding a desired safety distance to their leaders. They will form a chain of leader-follower-follower-follower continuing with followers until the desired amount of robots are in the formation. This formation is named duckling since this will be as a family of ducks where the children follow directly after their parents and does not care about anything on their way.

The echelon formation is a branch from the duckling formation where, instead of a direct pursuit, the followers are given a offset from the leader thus spanning the formation. This offset can be given in different ways i.e. with a desired difference in distance in a fixed angle from the leader. In the same way as the duckling formation this also has the opportunity to expand with a desired amount of robots in the formation that follows their respective leaders.

\section{Strategy 3: Multi Path Following}
Multi paths, or individual paths, for the vessels is where the agents follow their respective paths with respect to time along the path. This needs to make a formation for the vessels. They might only need to know from each other how far they are on their respective paths, such that every agent has a time depending factor on the path. If every agent fulfils its time dependency, this ensures, also by design of the path, that no collision will occur. The paths for the individual agents can be designed from the trajectory of the leader with an offset both in distance in a direction and a time. The time dependencies can make the following agents lack in time thus make this formation as a echelon formation, and the distance with a direction spans the formation itself. The different positions for the $i$th agent will then need to fulfil that the position is the desired position at a specific time
\begin{align}
\eta_d(t) = \eta(t)
\end{align}
where the error in position needs to be zero, $e_\eta(t) = \eta_d(t) - \eta(t) = 0$. The same needs to be fulfilled for the heading of the agents
\begin{align}
\psi_d(t) = \psi(t)
\end{align}
where the error in heading needs to be zero, $e_\psi(t) = \psi_d(t) - \psi(t) = 0$. When these two parameters given at the path are fulfilled the group of agents are said to be in formation. This formation strategy will not be tested as it did not get the highest rating from the analysis in the decision table \ref{tab:decision-matrix}. Though it can be a preferable way of designing the formation if the path can be generated to the specific case.
