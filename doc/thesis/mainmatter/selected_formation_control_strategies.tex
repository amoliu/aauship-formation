\chapter{Selected Formation Control Strategies}
\label{ch:selformctrl}
\head{This chapter investigates the previously mentioned formation control \todo{ref to seciton here} aspects with focus in three different types of formation control. These three types are analysed with focus on the specific task applied at the Port of Aalborg, afterwards simulated to be able to conclude which of these are the most beneficial to implement at the AAUSHIP fleet.}


\section{Relevant Characteristics}
To have a tool to determine what strategy is the best suited for the
problem described in \vref{sc:mission}, some parameters that describes
different characteristics of a strategy is needed. Listed here is a
description of each key parameter, also mentioning what is desired for
the mission at hand.

\begin{description}
\item[Communication] The communication requirements should be a measure of how much bandwidth is used. If the communication bandwidth usage is low it should be rated as a good thing, since the low bandwidth implies that the communication can be performed more easy and the scalability will become less complicated. It is not of importance if the communication breaks a link of the agents, this will be discussed in the individual formation control categories.

\item[Control architecture] The control architecture is a weighting of the complexity of the control structure. A high weighting will mean that the complexity is low and thus less complicated to implement. It will also be combined with a relation to the type of controller, where a more simple controller can be better if the overall objective can be accomplished. If the control can be done with a less complicated controller, i.e. a \ac{LOS} controller instead of some a more complex controller like \ac{MPC}, this can be preferred.

\item[Obstacle avoidance] The obstacle avoidance does not have a specific rating in the use case of the AAUSHIP due to the assumption of a open water manoeuvring. This means that the weighting of this should be taken as neutral and the implementation will be further work. Yet it is described in the specific control strategies because the implementation of this can be useful in the further work of the AAUSHIP projects. The obstacles can be known pre mission or they can be detected during mission and avoided and the task of the avoidance will take different shapes dependent of the specific task.

\item[Transients] The transients will be a weighting of how well the formation can make a turn and meanwhile keep the formation relative to the path. If the formation are unable to track the path during a turn it will be weighted low and will be the primary weighting. If the formation are able to track the path, but the formation will deviate a little, this will also be a down rating but not as much as if it is unable to track the path. Some formation problems during transients can be solved by the generation of the path but this will result as a limitation to the path generation.

\item[Scalability] The weighting of the scaling is done on two criteria. It is a combination of the structure of the complete formation and a estimate of the bandwidth usage from the higher need of communication. This weighting will be a summed weighting, thus both bandwidth and formation complexity is weighted equally. A rather simple expansion of the formation structure will be preferable, and a relative low bandwidth usage is good. If only communication from a single leader to one respective follower is required, this will be a good thing because if will half the bandwidth usage.
\end{description}

There are also some parameters that is hard to tell by the principle
of the method, which is listed below. These can be used to compare the
strategies after implementation.
\begin{description}
\item[Preparation time] This is related to the mission setup time. How
	much manual labour is needed to prepare the agents for a mission?
	This could for instance be trajectory generation and how the
	specific pattern needs to be generated. If the formation control strategy is chosen such that the agents needs to be close to the specific formation this will also take preparation time to place them relative to the desired formation.
\item[Time] The mission time for covering the area in a boustrophedon
	pattern. This will be from the start of the mission to a complete mission.
\item[Energy efficiency] This is a theoretical measure of the
	formation efficiency. The energy efficiency of any autonomous systems is relevant because it limits how long the autonomous system can operate autonomously, given that it cannot easily recharge or is fatal for the mission. This means that optimising the energy efficiency is desirable.
\item[Wear and tear] Is the controller aggressive? Aggressive
	controllers is known to make more wear and tear on the actuators. So
	if the strategy can be chosen to minimize the wear and tear this
	could be of benefit to the hardware.
\end{description}



%% Oversigt af control stats

\section{Methods}

\subsection{Direct LOS guidance}
The \ac{LOS} guidance makes the basis for a formation where one leader has a single follower connected and this follower has another single follower. \todo{Her skal nok stå en overordnet beskrivelse..}
\subsubsection{Duckling formation}
\label{sc:duckling}
This strategy takes the rise in a duckling or snake formation. In principle is this a leader that has a follower that has a follower and so on. Thereby will the formation take shape as a snake, when the leader takes off and the followers keep the line of sight to their respective leaders. The communication needed in the case of followers only follow one leader, and the communication only works in one direction, then the bandwidth usage is minimal. This implies that the communication it low and simple which is preferable. Similar to the communication will the control architecture become of the simple type where each agent only need to have a \ac{LOS} controller, such that they track their respective leaders. As such is obstacle avoidance not a native implementation of the formation, but can be implemented for the main leader of the formation. By doing this arises a problem of transients where the followers will 'cut corners'. They will directly pursuit their respective leaders, such that the formation will take sharper corners than the main leader. Although the formation does not perform well it has the advantage that the scalability is simple. When linking the formation in the duckling formation a new follower can be attached at the last agent and a new follower to this agent and so on. This implies that, in theory, an infinite tail of followers can be attached.

\subsubsection{Echelon formation}
The echelon formation is a branch from the duckling formation. In this formation the followers to the leader is not in a tail of the leader, but is offset into a echelon formation. The principle being the simple \ac{LOS} controller and the communication as followers to their respective leaders still apply in this branch. This implies that the problem statements from section \ref{sc:duckling} also applies to the echelon formation, and the main difference is the shape of the formation.
\subsubsection{Single \ac{FRP} with path}
\subsubsection{Multiple \ac{FRP} with path}

\subsection{Precomputed individual paths}

\subsubsection{Full communication}
\subsubsection{Limited communication}
\subsubsection{No communication}

\subsection{Potential field}
\subsubsection{Full communication}
\subsubsection{Limited communication}
\subsubsection{No communication}

\begin{itemize}
\item \textbf{Formation via precomputed time parametrised individual paths}
	\begin{itemize}
	\item \textbf{Adaptive formation. Is the time goal reached at each time step for all agents?}\\
	The implementation of this strategy needs to know for each time step for each agent when the goal is. By doing this it is possible to make the agents be in formation at all times because they have individual positions at their respective paths to specific times. If one of the agents does not reach its position in time, the rest of the agents should stop, or slow down, such that the missing agent has the possibility to catch up. This will change the 'time goals' for the rest of the agents, but the agents will keep the formation.
	\item \textbf{No information shared between agents}\\
	If no information is shared between the agents it is assumed that they do not experience any faults. If one of the agents has a fault, that it changes course or slows down, it will make the formation to fail. The individual agents will still follow their respective paths, and therefore reach their end goals, but not at the same time.
	\end{itemize}
\item \textbf{One \ac{FRP} with agents relative to this}\\
This method only makes use of one path for the whole formation. The agents in the setup will be defined with relative positions to a \ac{FRP} at each time step, such that the formation stays rigid and the arms from the \ac{FRP} to the agents are constants. This setup will work as a virtual leader in a rigid formation where only the point on the path (the \ac{FRP}) has information to the agents around it. The individual agents will not be able to communicate, due to the rigid formation.
\item \textbf{Multiple \ac{FRP}s where a formation follows another formation}\\
When more agents are present it could be of benefit to have these in individual subformations. This can be done by defining multiple \ac{FRP}s such that the formations are made relative to these points. Then it is the \ac{FRP}s that needs to be in formation and utilize their informations as formations follow the formation in front. This could be beneficial if a given formation needs to cover a specific area and the rest of the formation needs to continue. This is a tree like leader follower hierachy.
\todo{Det her er en MEGET vag formulering, men jeg er på vej ud af doeren.}
\item \textbf{One minima potential field formation}\\
Within the use of potential fields there can be applied on various ways. One way is to determine the formation based on one minima potential field and maxima at the agents. By different designs of these potential fields one can determine how the specific formation should be. One example could be with three agents where their potential maxima are equal both in magnitude and radius. If these three agents are placed near a single potential minima, they would go in direction of the steepest descent, which would be to the centre of the minima. Meanwhile would the agent's local maxima repulse each other which would result in a equally distributed formation around the potential minima, making a triangle formation around the minima.
This design can vary dependent on how the minima are chosen and also how the magnitude of the agent's potential fields are chosen. If multiple agents are added to the setup, and under the same assumptions as before, they would still distribute evenly around the potential minima, but do the more agents they would make a circle formation. \todo{This method cannot contorl the heading of the formation}
\item \textbf{Multiple minima potential field formation}\\
If an more uneven distribution is needed it could be of benefit to implement more local minima. These are not the minima that the formation should follow, but a higher level potential field, only to define the formation. This could be made of several local minima i.e. one for each agent if the triangle formation is performed. If a more complex formation is needed it would be impossible to make from potential fields if only one local minima could be used, as described in the item above. If a line formation is preferable it would need more local minima to make the distribution even.
\item \textbf{Non strict potential field with rain gutter principle}\\
The underlying potential field can be used when designing the path for the formation. This is done from the same principle of potential fields, but instead of the traditional fields that attract and repel evenly around a point, it can be designed as a rain gutter, which runs from the high magnitude to the lowest magnitude. This will make the formation run in the direction of the steepest descent, and therefore the path can be designed using this. If the formation is allowed to diverge from the path, and the requirement is not so strict, one can design the rain gutter more flat, like a 'u' instead of a 'v', where the 'v' would make the formation stay more precise on the determined path.
\item \textbf{Null Space Behaviour}

\end{itemize}

%% Approaches
\section{Strategy 1: Potential Field Formation}
\label{sc:potential-fields}
% Papers of interrest to this topic
% A Potential Field Based Approach to Multi-Robot Manipulation \citep{pfmrm}\\
% Formation Control with Configuration Space Constraints, \citep{fccsc}\\
% UAV Formation Flight using 3D Potential Field, \citep{UAVff3dpf}\\


\subsection{Potentialfield}
In potential fields it is commonly used that a multi-robot group should move in an environment, either with or without obstacles, some papers are \citep{pfmrm}, \citep{fccsc} and \citep{UAVff3dpf}. In the environment there need to be specified some relative measurements, or states, such that the robots are capable to manoeuvre relative to something. This reference, that the robots should manoeuvre with, is in this case the potential fields. When looking at robots that should keep distance from each other (in a formation), or should move to a desired position (trajectory tracking), the implementation of potential fields come in handy.

The potential field approach can be utilised or implemented in
different ways when taking about formation control. Potential fields
is often used for cooperative control where the formation between
agents is not as important, but rather have the goal of moving agetns into formation and then make groups of
agents move from one point to another \citep{5504176}. So to
use the potential field mindset it is important to define how these
fields works together to achieve some form of formation that can be
moved after a predefined path. 

Some ideas is to use an individual potential field seen from each
agent and combine this with the other agent's local potential field.
Here the potential field has local minima defining the desired
formation. In addition to this there is also a need to define how this
formation can be moved in a predefined path.

Another idea is to use the same potential field, that is a global and
common time varying potential field to define the formation, but this
will require that the agents are already near their formation, such
that every agent will approach their separate local minimum. This can
be used as a leader follower concept where the leader is defining this
global potential field.

The potential fields can be defined from attraction- and repulsion forces, dependent on if the robots need to move toward or away from a target, described in \citep{pfmrm}, where potential fields are utilized with a virtual structure. When applying a virtual structure the group of robots follow a virtual leader and not another robot relative to the formation. The potential functions can now be used as forces between the agents to repel them, such that they do not collide, which can be called inter-vehicle forces. The potential functions are also used as forces around the virtual leader such that the agents gets as close as possible to the virtual leader without ever reaching it, which can be called virtual leader forces. This will make a defined formation around the virtual leader \citep{1655803}. The repelling forces can be utilized in different ways and with advantage used with obstacle avoidance, where these need to contribute with a repelling force to the potential field, making agents diverge from these objects. A great advantage to the virtual leader is that the potential forces can be used to keep the formation and then only generate the trajectory for the virtual leader and not for every individual agent \citep{1655803}. The virtual can also be an agent as the reference point in the formation but is mostly thought of as the reference point for all agents in the formation.



\section{Strategy 2: Leader-Follower}

The leader follower principle is used as a higher order principle of how to navigate a group of robots. The principle is that a leader is defined to lead the group of robots in an environment relative to a trajectory. Instead of all the individual robots track their respective trajectories, only the leader follows a trajectory and the following robots keep their position relative to the leader position.

The way that the followers maintain their position can be done in different ways e.g. potential fields \citep{pfmrm}, behavioural methods as Null Space Based behavioural methods \citep{arrichiello2006formation} or versions of a direct \ac{LOS} algorithm. With the focus in here the formation should be defined as a 'rigid' formation, where the follower's positions are defined as fixed distances from the leader. The term rigid is not a strict description in this case because the formation can vary a little all the time, but it is not flexible either \citep{976029}.

The positions should be defined individually for each of the followers to the position of the leader. This can be done through a formation constraint function, F($\eta$), which should be a strict convex function to ensure the formation constraint (the actual formation). This formation constraint function can be expressed in different ways, with the actual positions of the agents and the virtual leader or positions of the agents relative to the virtual leaders starting position.

From the decision table \ref{tab:decision-matrix} it can be seen that duckling formation and echelon formation have got a high rating. These are both branches of the leader-follower principle which often shows to be applicable in formation issues \citep{TKP04}, \citep{1013687}, \citep{976029}.

The duckling formation is a direct intercept algorithm, where the followers intercept their respective leaders and withholding a desired safety distance to their leaders. They will form a chain of leader-follower-follower-follower continuing with followers until the desired amount of robots are in the formation. This formation is named duckling since this will be as a family of ducks where the children follow directly after their parents and does not care about anything on their way.

The echelon formation is a branch from the duckling formation where, instead of a direct pursuit, the followers are given a offset from the leader thus spanning the formation. This offset can be given in different ways i.e. with a desired difference in distance in a fixed angle from the leader. In the same way as the duckling formation this also has the opportunity to expand with a desired amount of robots in the formation that follows their respective leaders.

\section{Strategy 3: Multi Path following}
Individual paths for the vessels where they follow their respective trajectories with respect to the time. This needs to make a formation for the vessels. They might only need to know from each other how far they are on their respective trajectories, Christoffer Thorvaldsen.

\section{Strategy 4: Null-Space-Based approach}
% Papers of interrest to this topic
Formation Control of Underactuated Surface Vessels using the
Null-Space-Based Behavioral Control \citep{arrichiello2006formation}\\
Behavioral Control for Multi-Robot Perimeter Patrol: A Finite State
Automata Approach \citep{marino2009behavioral}\\



\begin{sidewaystable}
\begin{tabular}{l|lll}
\toprule
\textbf{Approach} & \textbf{Communication} & \textbf{Control architecture} & \textbf{Obstacle avoidance} \\
\hline
\textbf{Leader follower:}&&&\\
Duckling with LOS& Simple - Good & Simple LOS - Good & None - Poor \\
Skewed duckling with LOS& Simple - Good & Simple LOS - Good & None - Poor \\
\textbf{Precomputed individual paths:}&&&\\
Full communication& High bw - Poor & Multi agent info - Poor & None - Poor \\
Limited communication& Middle bw - OK & Local agent info - OK & None - Poor \\
No communication& Low bw - Good & Single agent info - Good & None - Poor \\
Single \ac{FRP} with path& Low communication - Good & Simple - Good & None - Poor \\
Multiple \ac{FRP} with path& Middle communication - OK & More agents - Poor & None - Poor \\
\textbf{Potential field:}&&&\\
Full communication& High bw - Poor & Multi agent info - Poor & Natural - Good \\
Limited communication& Middle bw - OK & Local agent info - OK & Natural - Good \\
No communication& Low bw - Good & Single agent info - Good & Natural - Good\\
\bottomrule
 & \textbf{Transients} & \textbf{Scalability}\\
\hline
\textbf{Leader follower:}&&&\\
Duckling with LOS   & Choosing/Poor & Easy duplicate - Good\\
Skewed duckling with LOS & Choosing/Poor & Easy duplicate - Good\\
\textbf{Precomputed individual paths:}&&&\\
Full communication   & Choosing/Good & OK\\
Limited communication   & Choosing/Good & Good\\
No communication   & Choosing/Good & Good\\
Single \ac{FRP} with path   & Choosing/OK & OK\\
Multiple \ac{FRP} with path   & Choosing/Poor & OK\\
\textbf{Potential field:}&&&\\
Full communication    & Good & OK\\
Limited communication    & Good & Good\\
No communication    & Good & Good\\
\end{tabular}
\caption{Decision matrix for the formation strategies}
\label{tab:decision-matrix}
\end{sidewaystable}

