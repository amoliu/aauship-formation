\chapter{Simulation Model} \head{This section will describe the model
that is used for simulation of the system, as a replacement for
testing on the real ship.}

To make a model for simulation model, it is needed to emulate the real
sensor outputs with noise imposed onto the signals. Using a \ac{LTI}
state space model based on the unified model \vref{eq:totmodel} is
constructed as \vref{eq:ss} as defined by \citep[p. 175]{fossen}.

\begin{subequations} \begin{align} \dot x &=  A x + B u + E w \\ A &=
		\begin{bmatrix} 0 & I\\ -M^{-1}G & -M^{-1}D \end{bmatrix}, \quad B
		= \begin{bmatrix} 0 \\ M^{-1} \end{bmatrix}, \quad E =
		\begin{bmatrix} 0 \\ M^{-1} \end{bmatrix} \end{align}
	\label{eq:ss} \end{subequations}

The matrix $E$ describes the sea state and vector $w$ is the
represents the process noise.

\nomenclature{$x_n, y_n$}{Position in the \acs{NED}-frame, usually
computed from a \acs{GPS}}
\nomenclature{$a_x, a_y, a_z$}{Linear accelerations from accelerometer}
\nomenclature{$m_x, m_y, m_z$}{Magnetic field from magnetometer}
\nomenclature{$\omega_x, \omega_y, \omega_z$}{Angular velocity from rate gyro}
\nomenclature{$\psi$}{Heading of ship}

\section{Sensor Measurements to State Vector}
For the control system it is needed to convert the sensor measurements
to the system state vector, such that the control system can be
designed. Figure~\vref{fig:intermediate-calc} shows the computation
flow to determine this. It shall be noted that the \ac{GPS} and
\ac{IMU} blocks has the sensor noise in them.

Now that the state vector is present an state observer can be used to
filter the measurements.

\begin{figure}
	\centering
	\includesvg{intermediate-sensor-calculations-block}
	\caption{Block diagram over the computation of system states from
	raw sensor measurements.}
	\label{fig:intermediate-calc}
\end{figure}
