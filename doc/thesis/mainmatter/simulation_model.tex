\chapter{Simulation Model}
\label{ch:simulation-model}
\head{This chapter will describe the model
that is used for simulation of the system, as a replacement for
testing on the real ship.}

To make a model for simulation model, it is needed to emulate the real
sensor outputs with noise imposed onto the signals. Using a \ac{LTI}
state space model based on the unified model \vref{eq:totmodel} is
constructed as \vref{eq:ss} as defined by \citep[p. 175]{fossen}.

\begin{subequations}
\begin{align}
	\dot x &=  A x + B u + E w \\
	y &= H x + v \\
	A &=
	\begin{bmatrix}
		0 & I\\ -M^{-1}G & -M^{-1}D
	\end{bmatrix}, \quad
	B = 
	\begin{bmatrix}
		0 \\ M^{-1}
	\end{bmatrix}, \quad
	E =
	\begin{bmatrix}
		0 \\ M^{-1}
	\end{bmatrix}
\end{align}
\label{eq:ss}
\end{subequations}

The matrix $E$ describes the sea state and vector $w$ is the
represents the process noise, and $v$ the sensor noise. Both noise
vectors are zero-mean Gaussian white noise processes.

\nomenclature{$x_n, y_n$}{Position in the \acs{NED}-frame, usually
computed from a \acs{GPS}}
\nomenclature{$a_x, a_y, a_z$}{Linear accelerations from accelerometer}
\nomenclature{$m_x, m_y, m_z$}{Magnetic field from magnetometer}
\nomenclature{$\omega_x, \omega_y, \omega_z$}{Angular velocity from rate gyro}
\nomenclature{$\psi$}{Heading of ship}

\section{Sensor Measurements to State Vector}
For the control system it is needed to convert the sensor measurements
to the system state vector, such that the control system can be
designed. Figure~\vref{fig:intermediate-calc} shows the computation
flow to determine this. It shall be noted that the \ac{GPS} and
\ac{IMU} blocks has the sensor noise in them.

Now that the state vector is present an state observer can be used for
in i.e. a Kalman filter to reduce the noise.

\begin{figure}
	\centering
	\includesvg{intermediate-sensor-calculations-block}
	\caption{Block diagram over the computation of system states from
	raw sensor measurements.}
	\label{fig:intermediate-calc}
\end{figure}

Since the simulation is performed by iterating over the state space
model, it is needed to somehow get the variances from the sensors
modelled in the state space model. Because of the intermediate
computations described in the figure~\vref{fig:intermediate-calc} it
is not straight forward to add the sensor noise to the model, because
this noise is specified at the raw sensor measurements. So some way
has to be used to calculate the noise $v$.

A method is the simply set the inputs as the variance value and the
output will be the variance in the state vector. \todo{Is this true?}

A method is to set the sensor measurements to no movement values, and
only add the noise on the measurements. I all sensors were zero in
stagnation, then it would be enough to simulate this with only noise and
get the corresponding variance out. This is not the case, as the
magnetometer has a bias, just because it is situated in a constant
magnetic field and this is dependent on the attitude of the ship. So
to compute the variances on the state vector it is needed to make
multiple simulations where this bias is different also. A normalized
normal distribution of this should suffice. \todo{Is this good enough
or even correct? We have nonlinearities.}

\section{Kalman Filter design}

A \ac{KF} is one type of observer that can be applied to estimate the state vector. This is done to filter the measurements from the vessel and smooth these. If the measurements are too noisy, such that the vessel changes direction suddenly, but should be surging forward, a filter can predict the modelled direction and compare this to the measurement. This would work as a low pass filter and make the changes in the measurements smaller with respect to the model predictions.

The \ac{KF} comprises the deterministic part of the model which estimates the state vector. This is corrected by means of measurements, to make the final estimated state vector. The process is modelled as seen on figure ~\ref{fig:blockkf} where the \ac{KF} is included.

\begin{figure}
	\centering
	\includesvg{kf_on_sys}
	\caption{Block diagram of kalman filter estimating the state vector.}
	\label{fig:blockkf}
\end{figure}

The \ac{KF} prediction and update can be written as:
\begin{itemize}
\item Prediction
\begin{align}
\hat x_{k}^- &= \Phi_{k}\ \hat x_{k-1}^- + G_{k}\ u_{k-1}\nonumber\\
\hat z_{k}^- &= H_{k}\ \hat x_{k}^-\nonumber\\
P_{k}^- &= \Phi_{k}\ P_{k-1}^-\ \Phi_{k}^T + Q_{k}\nonumber
\end{align}

\item Update
\begin{align}
K_{k} &= P_{k}^- H_{k}^T(H_{k} P_{k}^- H_{k}^T+R_{k})^{-1}\nonumber\\
\hat x_{k}^+ &= \hat x_{k}^- + K_{k}(z_{k}-\hat z_{k}^-)\nonumber\\
P_{k}^+ &= (I-K_{k}\ H_{k})P_{k}^-\nonumber
\end{align}
\end{itemize}
where $P$ is the correlation matrix, $Q$ is a measure of how much the model is to be trusted and $R$ is a measure of how much the measurement are to be trusted.
