\chapter{Path Generation}
\label{ch:pathgen}
\head{A guidance system usually consists of a subsystem to generate
	paths for the trajectory it is desired to have the object follow.
	There exists multiple methodologies for this, also depending on the
mission purpose. A description of these methods are described in this
chapter with focus on pre-mission defined paths.}

For these kind of guidance systems, it is usually generated via some
form of human interface -- some more intelligent than others. For the
purpose of this thesis, the interest lies in the use for surveying
purposes, such that a path should be generated from some waypoints
that is generated from some polygonal area of interest.

The concept of this is to make run lines (lawn mover pattern) with
straight line segments, and the corners can be handled in a specific
way such that the physical constraints of the ship manoeuvring ability
is considered to generate feasible paths.

The here is basically two ways to consider a lawn mover pattern. The
one is the have the area of interest be covered only by straight line
segments and the end connected with a turning maneouver outside the
area of interest as show on figure~\vref{fig:lawn-straight-inside} or
with all manoeuvers constrained inside the area of interest as
illustrated on figure~\vref{fig:lawn-all-inside}.

\begin{figure}[htbp]
	\centering
	\subfloat[Only straight lines\label{fig:lawn-straight-inside}]{\includesvg{lawn_straight_inside}}
	\qquad
	\subfloat[All paths constrained\label{fig:lawn-all-inside}]{\includesvg{lawn_all_inside}}
	\caption{Comparison of two ways to cover an area with a lawn mower
	pattern.}
\end{figure}

\section{Dubins Path}
\todo{This chapter will depend on the formation control choosen.}


\missingfigure{Figure of Dubins path generated with fixed and angle
dependent circles}

\section{Improvement of Dubins Path with Clothoids}
The goal with using Clothoids (Euler Spirals) is the have a yank
continuous parametrized path.

\section{\acs{LOS} Guidance}
\acl{LOS} is one paradigm for a path following controller. It can be relised in different ways.

\section{Guidance System}
The guidance system is a system that takes the wishes from the
operator and converts those to some trajectory via a path generating
algorithm.


\begin{figure}[htbp]
	\centering
	\includesvg{guidance-overview}
	\caption{Overview of guidance system where a mission static waypoint
	database is used with a \ac{LOS} algorithm.}
\end{figure}


\begin{figure}[htbp]
	\centering
	\includesvg[width=\textwidth]{use-case-polygon}
	\caption{Comparison of two ways to cover an area with a lawn mower
	pattern.}
    \label{fig:use-case-polygon}
\end{figure}


Tesselation with quadrilaterals (polygons with four edges and four vertices) is a way to help divide a given polygonal shape into some shape where it is usefull to use the plough fure motion.

