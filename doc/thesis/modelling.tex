\chapter{Modelling}
\head{Within this chapter will different models be developed to make the actuation of the vessel more precise.}

\section{Added Mass Modelling}
Definition: Hydrodynamic added mass is defined as the mass added to a system due to an accelerating or decelerating body must move a volume of the surrounding fluid as it moves through it. To this is said that the object and fluid is not able to occupy the same physical space simultaneously.

\section{Rigid Body Modelling}
The rigid body is used to model the physics of the vessel. It is an idealization of the solid body from where the physical motions of the vessel are to be derived. From analysis of this can translational motion and rotational motion be derived, and by \citep{fossen} written in component form as:
\begin{align}
f^b_b &= [X,Y,Z]^T & &- \text{force through } o_b \text{ expressed in } \{b\}\\
m^b_b &= [K,M,N]^T & &- \text{moment about } o_b \text{ expressed in } \{b\}\\
v^b_{b/n} &= [u,v,w]^T & &- \text{linear velocity of } o_b \text{ relative } o_n \text{ expressed in } \{b\}\\
\omega^b_{b/n} &= [p,q,r]^T & &- \text{angular velocity of } {b} \text{ relative to } \{n\} \text{ expressed in } \{b\}\\
r^b_g &= [x_g,y_g,z_g]^T & &- \text{vector from } o_b \text{ to CG expressed in } \{b\}
\end{align}

\section{Disturbances}


\subsection{Linear}


\subsection{Non Linear}


\section{Total Model of Vessel}

\section{Identification of Hydrodynamic Derivatives}
The linear kinetic model \eqref{eq:MDt} consisting of the mass matrix $M$ and the damping matrix $D$, which has some coefficients that can be determined by tests.
\begin{align}
M \dot \nu + D \nu = \tau
\label{eq:MDt}
\end{align}\todo{Not entirely correct}

These coefficients can be determined in multiple ways. Often times for a ship design company they are able to use \ac{CFD} to determin e the coefficients, but theese application are often expensive and propriotary. So a third method to do this is to use use soem simple tests to do the approximations.

For the forward coefficients
\begin{align}
M_{11} \ddot x + D_{11} \dot x = 0
\end{align}

One can then solve for $D_{11}$
 
