\chapter{Modelling}
\head{Within this chapter will different models be developed to make the actuation of the vessel more precise.}

\section{Hydrodynamic Modelling}
Definition: Hydrodynamic added mass is defined as the mass added to a system due to an accelerating or decelerating body must move a volume of the surrounding fluid as it moves through it. To this is said that the object and fluid is not able to occupy the same physical space simultaneously.
\begin{align}
M_A \dot \nu_r + C_A(\nu_r)\nu_r + D(\nu_r)\nu_r = \tau
\label{eq:hydmodel}
\end{align}
where
\begin{align}
&M_A \text{is the added mass matrix from the system}\nonumber\\
&C_A \text{ is the added mass matrix due to coriolis}\nonumber\\
&D(\nu) \text{ is both the potential and viscous damping matrices}\nonumber\\
&\tau \text{ is control and propulsion forces}\nonumber\\
&\nu \text{ is the velocities of the vessel in all directions and moments}
\end{align}

\section{Rigid Body Modelling}
The rigid body is used to model the physics of the vessel. It is an idealization of the solid body from where the physical motions of the vessel are to be derived. From analysis of this can translational motion and rotational motion be derived, and by \citep{fossen} written in component form as:
\begin{align}
f^b_b &= [X,Y,Z]^T & &- \text{force through } o_b \text{ expressed in } \{b\}\\
m^b_b &= [K,M,N]^T & &- \text{moment about } o_b \text{ expressed in } \{b\}\\
v^b_{b/n} &= [u,v,w]^T & &- \text{linear velocity of } o_b \text{ relative } o_n \text{ expressed in } \{b\}\\
\omega^b_{b/n} &= [p,q,r]^T & &- \text{angular velocity of } {b} \text{ relative to } \{n\} \text{ expressed in } \{b\}\\
r^b_g &= [x_g,y_g,z_g]^T & &- \text{vector from } o_b \text{ to CG expressed in } \{b\}
\end{align}

\begin{align}
M_{RB} \dot \nu_r + C_{RB}(\nu_r)\nu_r = \tau_{RB}
\label{eq:rigidmodel}
\end{align}
where
\begin{align}
&M_{RB} \text{is the system inertia matrix}\nonumber\\
&C_{RB} \text{ is coriolis-centriopedal matrix}\nonumber\\
&\tau_{RB} \text{ is a lumped force combined of } \tau_{hyd} + \tau_{hs} + \tau_{wind} + \tau_{wave} + \tau\nonumber\\
&\quad \text{where}\nonumber\\
&\qquad \tau_{hyd} \text{ is the hydrodynamic force}\nonumber\\
&\qquad \tau_{hs} \text{ is the hydrostatic force}\nonumber\\
&\qquad \tau_{wind} \text{ is the wind force}\nonumber\\
&\qquad \tau_{wave} \text{ is the wave force}\nonumber\\
&\qquad \tau \text{ is the control and propulsion forces}\nonumber
\end{align}

\section{Total Model of Vessel}
\begin{align}
\underbrace{M_{RB} \dot \nu_r + C_{RB}(\nu_r)\nu_r}_{\text{rigid-body forces}} + \underbrace{M_A \dot \nu_r + C_A(\nu_r)\nu_r + D(\nu_r)\nu_r}_{\text{hydrodynamic forces}}  = \tau + \tau_{RB}
\label{eq:totmodel}
\end{align}

Delimitation:
Since the vessel within this project is of smaller scale, the $M_A$, $C_A$ and $C_{RB}$ from \ref{eq:hydmodel} and \ref{eq:rigidmodel} are neglected. $M_A$ is the added mass and is as a start omitted due to the tests needs to be made as an object moving through the water with some drag. If the model needs to be further improved in the process this is a place to start modelling. The coefficients of $M_A$ are rather inconvenient to determine without advanced equipment like a towing tank, where constant velocity can be applied and measure drag and more in all directions and moments. $C_A$ and $C_{RB}$ represents forces due to a rotation of the body frame, \{b\}, about the inertial frame, the NED frame. These are omitted as well due to the small vessel where the body frame is placed in a predefined local frame which acts as the NED frame. This reduces equation \ref{eq:totmodel} down to the following
\begin{align}
M_{RB} \dot \nu_r + D\nu_r = \tau_{RB} + \tau
\label{eq:reducedmodel}
\end{align}
The damping matrix which contains the coefficients of the drag is denoted the hydrodynamic damping matrix. This consists both of $D$ which is the linear damping matrix due to potential damping and possible skin friction with the water and $D_n(\nu_r)$ which is the nonlinear damping matrix due to quadratic damping and higher order terms.  This will, as a start be modelled as the linear part, being potential and viscous damping. As higher velocities will the nonlinear part become more dominant due to the quadratic terms of the velocity, thus is mostly used at faster vessels. The linear damping matrix $D$ contributes more at lower speed manoeuvring and stationkeeping. Therefore is the damping matrix $D$ used, and is expressed by ~\citep{fossen} for a 6 \ac{DOF} system to be
\begin{align}
D =-
\begin{bmatrix}
X_u & 0 & 0 & 0 & 0 & 0\\
0 & Y_v & 0 & Y_p & 0 & Y_r\\
0 & 0 & Z_w & 0 & Z_q & 0\\
0 & K_v & 0 & K_p & 0 & K_r\\
0 & 0 & M_w & 0 & M_q & 0\\
0 & N_v & 0 & N_p & 0 & N_r
\end{bmatrix}
\end{align}

The rigid-body system matrix of the vessel is given for a 6 \ac{DOF} system by ~\citep{fossen} as:
\begin{align}
M_{RB} =
\begin{bmatrix}
m\boldsymbol{I}_{3x3} & -m\boldsymbol{S}(r^b_g)\\
-m\boldsymbol{S}(r^b_g) & \boldsymbol{I}_b
\end{bmatrix}
=
\begin{bmatrix}
m & 0 & 0 & 0 & mz_g & -my_g\\
0 & m & 0 & -mz_g & 0 & mx_g\\
0 & 0 & m & my_g & -mx_g & 0\\
0 & -mz_g & my_g & I_x & -I_{xy} & -I_{xz}\\
mz_g & 0 & -mx_g & -I_{yx} & I_y & -I_{yz}\\
-my_g & mx_g & 0 & -I_{zx} & -I_{zy} & I_z
\end{bmatrix}
\end{align}

\section{Identification of Hydrodynamic Derivatives}
The linear kinetic model \eqref{eq:MDt} consisting of the mass matrix $M$ and the damping matrix $D$, which has some coefficients that can be determined by tests.
\begin{align}
M \dot \nu + D \nu = \tau
\label{eq:MDt}
\end{align}\todo{Not entirely correct}

These coefficients can be determined in multiple ways. Often times for a ship design company they are able to use \ac{CFD} to determin e the coefficients, but theese application are often expensive and propriotary. So a third method to do this is to use use soem simple tests to do the approximations.

For the forward coefficients
\begin{align}
M_{11} \ddot x + D_{11} \dot x = 0
\end{align}

One can then solve for $D_{11}$
 
