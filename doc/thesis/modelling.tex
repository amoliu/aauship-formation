\chapter{Modelling}
\head{Within this chapter will different models be developed to make the actuation of the vessel more precise.}

\section{Added Mass Modelling}
Definition: Hydrodynamic added mass is defined as the mass added to a system due to an accelerating or decelerating body must move a volume of the surrounding fluid as it moves through it. To this is said that the object and fluid is not able to occupy the same physical space simultaneously.

\section{Rigid Body Modelling}
The rigid body is used to model the physics of the vessel. It is an idealization of the solid body from where the physical motions of the vessel are to be derived. From analysis of this can translational motion and rotational motion be derived, and by \citep{fossen} written in component form as:
\begin{align}
f^b_b &= [X,Y,Z]^T & &- \text{force through } o_b \text{ expressed in } \{b\}\\
m^b_b &= [K,M,N]^T & &- \text{moment about } o_b \text{ expressed in } \{b\}\\
v^b_{b/n} &= [u,v,w]^T & &- \text{linear velocity of } o_b \text{ relative } o_n \text{ expressed in } \{b\}\\
\omega^b_{b/n} &= [p,q,r]^T & &- \text{angular velocity of } {b} \text{ relative to } \{n\} \text{ expressed in } \{b\}\\
r^b_g &= [x_g,y_g,z_g]^T & &- \text{vector from } o_b \text{ to CG expressed in } \{b\}
\end{align}

\section{Disturbances}


\subsection{Linear}


\subsection{Non Linear}


\section{Total Model of Vessel}


