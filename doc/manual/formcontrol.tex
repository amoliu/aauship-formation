\section{Formation control}
\head{This section will give a short introduction to formation control in general followed by some of the individual assignments which needs to be determined before the process of design can be started.}
The theory of formation control in general is widely applied. It can be applied in assignments regarding control of robots which needs to be placed relative to each other. Depending on the given task of the robots, and which type of robots are in focus, the formation can be given in different ways. The robots can also be of various types: Swarm robots, driving vehicles, helicopters, air planes, ships etc. which can both be manned and unmanned. 



\textbf{inline 475:}\\
g6606\\
58 euro\\
1217 kv\\
aksel 3.17mm\\
nom 11.1V max 14.8V\\

\textbf{compact 460z:}\\
g7741\\
46 euro\\
900 kv\\
aksel 4mm\\
nom 14.8V max 18.5V\\

\textbf{compact 465z:}\\
7772\\
64 euro\\
600 kv\\
aksel 5mm\\
nom 14.8V max 18.5V\\
